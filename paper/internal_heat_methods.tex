%\documentclass[iop]{emulateapj}
\documentclass[aps, pre, onecolumn, nofootinbib, notitlepage, groupedaddress, amsfonts, amssymb, amsmath, longbibliography]{revtex4-1}
\usepackage{graphicx}
\usepackage{hyperref}
\usepackage{xcolor}
\hypersetup{
    colorlinks,
    linkcolor={red!50!black},
    citecolor={blue!50!black},
    urlcolor={blue!80!black}
}
\usepackage{bm}
\usepackage{natbib}
\usepackage{longtable}
\LTcapwidth=0.87\textwidth

\newcommand{\Div}[1]{\ensuremath{\nabla\cdot\left( #1\right)}}
\newcommand{\angles}[1]{\ensuremath{\left\langle #1 \right\rangle}}
\newcommand{\grad}{\ensuremath{\nabla}}
\newcommand{\RB}{Rayleigh-B\'{e}nard }
\newcommand{\stressT}{\ensuremath{\bm{\bar{\bar{\Pi}}}}}
\newcommand{\lilstressT}{\ensuremath{\bm{\bar{\bar{\sigma}}}}}
\newcommand{\nrho}{\ensuremath{n_{\rho}}}
\newcommand{\approptoinn}[2]{\mathrel{\vcenter{
	\offinterlineskip\halign{\hfil$##$\cr
	#1\propto\cr\noalign{\kern2pt}#1\sim\cr\noalign{\kern-2pt}}}}}

\newcommand{\appropto}{\mathpalette\approptoinn\relax}

\newcommand\mnras{{MNRAS}}%

\begin{document}
\author{Evan H. Anders}
\affiliation{Dept. Astrophysical \& Planetary Sciences, University of Colorado -- Boulder, Boulder, CO 80309, USA}
\affiliation{Laboratory for Atmospheric and Space Physics, Boulder, CO 80303, USA}
\author{Benjamin P. Brown}
\affiliation{Dept. Astrophysical \& Planetary Sciences, University of Colorado -- Boulder, Boulder, CO 80309, USA}
\affiliation{Laboratory for Atmospheric and Space Physics, Boulder, CO 80303, USA}
\title{Internally heated, stratified, compressible convection}

\begin{abstract}
An abstract will go here eventually
\end{abstract}
\maketitle


\section{Introduction}
\label{sec:intro}
\begin{enumerate}
\item Importance of internal heating in natural convective systems
\item Shape of internal heating in something like Sun
\item Studies in Rayleigh-Benard
\item Goals of this paper:
\begin{enumerate}
\item Show how to set up internally heated polytropes, with stable layers (or without)
\item Show how the resultant stratification depends on combo of lower flux + internal heating profile choices
\item Show how stable layer affects surface power spectrum
\end{enumerate}
\end{enumerate}

\section{Experiment} 
\label{sec:experiment}
We choose a parameter, $\epsilon$, which sets the initial scale of gravity,
\begin{equation}
g \equiv m_{ad} + 1 - \epsilon,
\end{equation}
where $m_{ad} \equiv (\gamma - 1)^{-1}$, and $\gamma = c_P / c_V = 5/3$.  The atmosphere
has a constant internal heating term, such that the energy equation looks like
\begin{equation}
\frac{\partial}{\partial t}\left(\left[\rho\frac{|\bm{u}|^2}{2} + c_v T + \phi\right]\right)
+ \grad\cdot\left(\bm{F}_{\text{conv}} + \bm{F}_{\text{cond}}\right) = \kappa H,
\end{equation}
where $H$ is an internal heating term.  In the initial, static, conductive state, the
conductive flux must balance the internal heating term,
\begin{equation}
\grad\cdot\left(-\kappa\grad T_{c}\right) = \kappa H,
\end{equation}
A useful form of the initial temperature gradient in such a system is
\begin{equation}
\grad T_{c} = - 1 + H (L_z - z).
\end{equation}
This initial conductive solution is \emph{not} the adiabatic solution to this atmosphere,
\begin{equation}
\grad T_{ad} = -\frac{g}{c_P} = -1 + \frac{\epsilon}{c_P}.
\end{equation}

The initial \emph{superadiabatic temperature gradient} is an important quantity, as it
determines whether or not the atmosphere is stable to convection.  This gradient
in these systems is
\begin{equation}
\grad T_{c} - \grad T_{ad} = H(L_z - z) - \frac{\epsilon}{c_P}.
\end{equation}
When this quantity is zero, the atmosphere is carrying the maximum amount of flux
that conductivity can carry.  When this value is less than zero, the atmosphere
requires convection to carry the flux.  Solving for a generalized zero-point, we
find that
\begin{equation}
z_{\text{cross}} \equiv L_z - \frac{\epsilon}{H c_P}.
\end{equation}
In order for the atmosphere to have any region that is unstable to convection,
we require that $0 < \epsilon / ( H c_P ) < L_z$.  In the case that 
$\epsilon / ( H c_P ) < L_z$, the whole atmosphere is unstable to convection.
Further, the anticipated length scale of the convecting region of the atmosphere
is the full atmosphere above $z_{\text{cross}}$, or
\begin{equation}
d_{\text{conv}} = L_z - z_{\text{cross}} = \frac{\epsilon}{H c_P}
\end{equation}

I think there are two natural control parameters of this system:
\begin{enumerate}
\item $f \equiv z_{\text{cross}} / L_z$, the fraction of the atmosphere that is stable
($f = 0$ means the full atmosphere convects, $f = 0.5$ means that the upper half convects, etc).
This \emph{should} be the parameter that determines the depth of the stable layer.
\item $\epsilon$, the deviation of the initial temperature gradient (at the top of the atmosphere)
from the adiabatic temperature gradient. This \emph{should} be the parameter that determines
the depth of the stable layer.
\end{enumerate}
By specifying these two parameters, we retrieve the magnitude of the internal heating,
$ f = 1 - \epsilon / (H c_P L_z)$
\begin{equation}
H = \frac{\epsilon}{L_z c_P (1 - f)}.
\end{equation}

There are still three parameters in these systems that need to be figured out:
\begin{enumerate}
\item Density stratification
\item thermal diffusivity
\item viscous diffusvitiy
\end{enumerate}
One of these parameter is free,
\begin{equation}
\text{Pr} = \frac{\nu}{\chi}.
\end{equation}
We're not going to mess with that.

There will be some type of Rayleigh number in these systems,
\begin{equation}
\text{Ra} = \frac{\text{stuff that comes from the atmosphere}}{\nu \chi},
\end{equation}
but I need to think a little more and spend some time with the equations to
figure out what this is.

And then as for the stratification....well, I need to figure out what $n_{\rho}$ I'm specifying.
The $n_\rho$ of the corresponding adiabatic polytrope?  

\section{Results \& Discussion}
\label{sec:results}

\bibliography{./biblio.bib}
\end{document}

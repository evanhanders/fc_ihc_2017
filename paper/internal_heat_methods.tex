%\documentclass[iop]{emulateapj}
\documentclass[aps, pre, onecolumn, nofootinbib, notitlepage, groupedaddress, amsfonts, amssymb, amsmath, longbibliography]{revtex4-1}
\usepackage{graphicx}
\usepackage{hyperref}
\usepackage{xcolor}
\hypersetup{
    colorlinks,
    linkcolor={red!50!black},
    citecolor={blue!50!black},
    urlcolor={blue!80!black}
}
\usepackage{bm}
\usepackage{natbib}
\usepackage{longtable}
\LTcapwidth=0.87\textwidth

\newcommand{\Div}[1]{\ensuremath{\nabla\cdot\left( #1\right)}}
\newcommand{\angles}[1]{\ensuremath{\left\langle #1 \right\rangle}}
\newcommand{\grad}{\ensuremath{\nabla}}
\newcommand{\RB}{Rayleigh-B\'{e}nard }
\newcommand{\stressT}{\ensuremath{\bm{\bar{\bar{\Pi}}}}}
\newcommand{\lilstressT}{\ensuremath{\bm{\bar{\bar{\sigma}}}}}
\newcommand{\nrho}{\ensuremath{n_{\rho}}}
\newcommand{\approptoinn}[2]{\mathrel{\vcenter{
	\offinterlineskip\halign{\hfil$##$\cr
	#1\propto\cr\noalign{\kern2pt}#1\sim\cr\noalign{\kern-2pt}}}}}

\newcommand{\appropto}{\mathpalette\approptoinn\relax}

\newcommand\mnras{{MNRAS}}%

\begin{document}
\author{Evan H. Anders}
\affiliation{Dept. Astrophysical \& Planetary Sciences, University of Colorado -- Boulder, Boulder, CO 80309, USA}
\affiliation{Laboratory for Atmospheric and Space Physics, Boulder, CO 80303, USA}
\author{Benjamin P. Brown}
\affiliation{Dept. Astrophysical \& Planetary Sciences, University of Colorado -- Boulder, Boulder, CO 80309, USA}
\affiliation{Laboratory for Atmospheric and Space Physics, Boulder, CO 80303, USA}
\title{Internally heated, stratified, compressible convection}

\begin{abstract}
An abstract will go here eventually
\end{abstract}
\maketitle


\section{Introduction}
\label{sec:intro}
\begin{enumerate}
\item Importance of internal heating in natural convective systems
\item Shape of internal heating in something like Sun
\item Studies in Rayleigh-Benard
\item Goals of this paper:
\begin{enumerate}
\item Show how to set up internally heated polytropes, with stable layers (or without)
\item Show how the resultant stratification depends on combo of lower flux + internal heating profile choices
\item Show how stable layer affects surface power spectrum
\end{enumerate}
\end{enumerate}

\section{Experiment} 
\label{sec:experiment}

\subsection{General system info and temperature profile}
We choose a parameter, $\epsilon$, which sets the initial scale of gravity,
\begin{equation}
g \equiv m_{ad} + 1 - \epsilon,
\end{equation}
where $m_{ad} \equiv (\gamma - 1)^{-1}$, and $\gamma = c_P / c_V = 5/3$.  The atmosphere
has a constant internal heating term, such that the energy equation looks like
\begin{equation}
\frac{\partial}{\partial t}\left(\left[\rho\frac{|\bm{u}|^2}{2} + c_v T + \phi\right]\right)
+ \grad\cdot\left(\bm{F}_{\text{conv}} + \bm{F}_{\text{cond}}\right) = \kappa H,
\end{equation}
where $H$ is an internal heating term.  In the initial, static, conductive state, the
conductive flux must balance the internal heating term,
\begin{equation}
\grad\cdot\left(-\kappa\grad T_{c}\right) = \kappa H,
\end{equation}
A useful form of the initial temperature gradient in such a system is
\begin{equation}
\grad T_{c} = - 1 + H (L_z - z).
\end{equation}
This initial conductive solution is \emph{not} the adiabatic solution to this atmosphere,
\begin{equation}
\grad T_{ad} = -\frac{g}{c_P} = -1 + \frac{\epsilon}{c_P}.
\end{equation}
If we integrate the conductive solution, our initial temperature profile is
$$
T_c(z) = (H L_z - 1) z - \frac{H}{2}z^2 + C
$$
and if we want $T_c(L_z) = T_t$,
\begin{equation}
T_c(z) = -\frac{H}{2}z^2 - (1 - H L_z) z + \left(T_t - \frac{H}{2}L_z^2 + L_z\right).
\end{equation}
So our initial temperature gradient is now quadratic.

The initial \emph{superadiabatic temperature gradient} is an important quantity, as it
determines whether or not the atmosphere is stable to convection.  This gradient
in these systems is
\begin{equation}
\grad T_{c} - \grad T_{ad} = H(L_z - z) - \frac{\epsilon}{c_P}.
\end{equation}
When this quantity is zero, the atmosphere is carrying the maximum amount of flux
that conductivity can carry.  When this value is less than zero, the atmosphere
requires convection to carry the flux.  Solving for a generalized zero-point, we
find that
\begin{equation}
z_{\text{cross}} \equiv L_z - \frac{\epsilon}{H c_P}.
\end{equation}
In order for the atmosphere to have any region that is unstable to convection,
we require that $0 < \epsilon / ( H c_P ) < L_z$.  In the case that 
$\epsilon / ( H c_P ) < L_z$, the whole atmosphere is unstable to convection.
Further, the anticipated length scale of the convecting region of the atmosphere
is the full atmosphere above $z_{\text{cross}}$, or
\begin{equation}
d_{\text{conv}} = L_z - z_{\text{cross}} = \frac{\epsilon}{H c_P}
\end{equation}

I think there are two natural control parameters of this system:
\begin{enumerate}
\item $f \equiv z_{\text{cross}} / L_z$, the fraction of the atmosphere that is stable
($f = 0$ means the full atmosphere convects, $f = 0.5$ means that the upper half convects, etc).
This \emph{should} be the parameter that determines the depth of the stable layer.
\item $\epsilon$, the deviation of the initial temperature gradient (at the top of the atmosphere)
from the adiabatic temperature gradient. This \emph{should} be the parameter that determines
the depth of the stable layer.
\end{enumerate}
By specifying these two parameters, we retrieve the magnitude of the internal heating,
$ f = 1 - \epsilon / (H c_P L_z)$
\begin{equation}
H = \frac{\epsilon}{L_z c_P (1 - f)}.
\end{equation}

\subsection{Initial density stratification}

The initial density profile is set by hydrostatic equilibrium,
\begin{equation}
\grad\ln\rho_c = -\frac{g + \grad T_c}{T_c},
\end{equation}
which we numerically integrate to find $\ln\rho_c$ and $\rho_c$.  By the way, the reason
we do this numerically is because, according to wolfram,
$$
\int\frac{A + Bz}{C + Dz + Ez^2} = \frac{B}{2E}\ln\left(C + zD + Ez^2\right)
- \frac{(BD - 2AE)}{E\sqrt{4CE - D^2}}\tan^{-1}\left(\frac{D + 2 Ez}{\sqrt{4CE - D^2}}\right)
$$
And for this,
$$
A = -\left(g - 1 + HLz\right)\qquad B = H \qquad C = \left(T_t - \frac{H}{2}L_z^2 + L_z\right)
\qquad D = (HL_z - 1) \qquad E = -\frac{H}{2}
$$
which is to say that $E = -B/2$ and $A = -(D + g)$.  Simplifying the density profile
form a bit,
$$
\ln\rho_c = -\ln(T_c) - \frac{-2D - 2A}{\sqrt{4CE - D^2}}\tan^{-1}\left(\frac{D + 2Ez}{\sqrt{4CE - D^2}}\right)
$$
So that first term, the $\ln T_c$ term, cancels nicely.  the other term is kind of a mess, and to see
it more clearly, note that
$$
A + D = -g \qquad 4CE - D^2 = -2H\left(T_t - \frac{H}{2}L_z^2 + L_z\right) - (H^2 L_z^2 + 1 - 2HL_z)
= - 2(HT_t + 1),
$$
and that's sort of problematic, and also imaginary once we take the square root of it.  Note also
that $D + 2Ez = HL_z - 1 - Hz = H(L_z - z) - 1$, so the total function is
$$
\ln\rho_c = -\ln(T_c) - \frac{2g}{\sqrt{-2(HT_t + 1)}}\tan^{-1}\left(\frac{H(L_z - z) - 1}{\sqrt{-2(HT_t + 1)}}\right)
+ const
$$

Acknowledging that we're dealing with nasty imaginary numbers, note that $\tan^{-1}(A\sqrt{-1}) = i \tanh^{-1}(A)$,
so this function is really
\begin{equation}
\ln\rho_c = -\ln(T_c) - \frac{2g}{\sqrt{2(HT_t + 1)}}\tanh^{-1}\left(\frac{H(L_z - z) - 1}{\sqrt{2(H T_t + 1)}}\right) + C
\end{equation}
To make things easier for myself, I will define $\chi \equiv \sqrt{2(HT_t + 1)}$, and then I will note
$\tanh^{-1}$ has a kinda cool formula,
$$
\tanh^{-1}(A) = \frac{1}{2}\left(\ln(A + 1) - \ln(1 - A)\right) = \frac{1}{2}\ln\left(\frac{A + 1}{1 - A}\right)
= \frac{1}{2}\ln\left(\frac{B + C}{C - B}\right),
$$
where $A \equiv B / C$, so
$$
\tanh^{-1}\left(\frac{\grad T_c}{\chi}\right) =
\frac{1}{2}\ln\left(\frac{\grad T_c + \chi}{\chi - \grad T_c}\right)
$$
So our function looks like
\begin{equation}
\ln\rho_c = -\ln(T_c) - \frac{g}{\chi}\ln\left(\frac{\chi + \grad T_c}{\chi - \grad T_c}\right) + C
\end{equation}
At this point, I'm going to \emph{assume that $\rho_t = 1$, so $\ln\rho_c(L_z) = 0$}, and while
this kills some of our generality, it makes life a bit better.  also recall that we
have specified $\grad T_c = -1$ at the top of the atmosphere, so
$$
C = \ln(T_t) + \frac{g}{\chi}\ln\left(\frac{\chi - 1}{\chi + 1}\right)
$$
And in the end, our initial density profile is specified by
\begin{equation}
\ln\rho_c = -\ln\left(\frac{T_c}{T_t}\right) - \frac{g}{\chi}\ln\left(
\frac{\grad T_c + \chi}{\chi- \grad T_c}\cdot
\frac{\chi + 1}{\chi - 1}
\right)
\end{equation}

There are still three parameters in these systems that need to be figured out:
\begin{enumerate}
\item Density stratification
\item thermal diffusivity
\item viscous diffusvitiy
\end{enumerate}
One of these parameter is free,
\begin{equation}
\text{Pr} = \frac{\nu}{\chi}.
\end{equation}
We're not going to mess with that.

There will be some type of Rayleigh number in these systems,
\begin{equation}
\text{Ra} = \frac{\text{stuff that comes from the atmosphere}}{\nu \chi},
\end{equation}
but I need to think a little more and spend some time with the equations to
figure out what this is.

And then as for the stratification....well, I need to figure out what $n_{\rho}$ I'm specifying.
The $n_\rho$ of the corresponding adiabatic polytrope?  

\section{Results \& Discussion}
\label{sec:results}

\bibliography{./biblio.bib}
\end{document}

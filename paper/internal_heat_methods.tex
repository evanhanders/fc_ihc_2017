\documentclass[twocolumn]{aastex61}
\usepackage{graphicx}
\usepackage{amsmath}
\usepackage{amssymb}
\usepackage{amsfonts}
\usepackage{hyperref}
\usepackage{xcolor}
\hypersetup{
    colorlinks,
    linkcolor={red!50!black},
    citecolor={blue!50!black},
    urlcolor={blue!80!black}
}
\usepackage{bm}
\usepackage{natbib}
\usepackage{longtable}
\LTcapwidth=0.87\textwidth

\newcommand{\Div}[1]{\ensuremath{\nabla\cdot\left( #1\right)}}
\newcommand{\angles}[1]{\ensuremath{\left\langle #1 \right\rangle}}
\newcommand{\grad}{\ensuremath{\nabla}}
\newcommand{\RB}{Rayleigh-B\'{e}nard }
\newcommand{\stressT}{\ensuremath{\bm{\bar{\bar{\Pi}}}}}
\newcommand{\lilstressT}{\ensuremath{\bm{\bar{\bar{\sigma}}}}}
\newcommand{\nrho}{\ensuremath{n_{\rho}}}
\newcommand{\approptoinn}[2]{\mathrel{\vcenter{
	\offinterlineskip\halign{\hfil$##$\cr
	#1\propto\cr\noalign{\kern2pt}#1\sim\cr\noalign{\kern-2pt}}}}}

\newcommand{\appropto}{\mathpalette\approptoinn\relax}

\begin{document}
\author{Evan H. Anders}
\affiliation{Dept. Astrophysical \& Planetary Sciences, University of Colorado -- Boulder, Boulder, CO 80309, USA}
\affiliation{Laboratory for Atmospheric and Space Physics, Boulder, CO 80303, USA}
\author{Benjamin P. Brown}
\affiliation{Dept. Astrophysical \& Planetary Sciences, University of Colorado -- Boulder, Boulder, CO 80309, USA}
\affiliation{Laboratory for Atmospheric and Space Physics, Boulder, CO 80303, USA}
\title{Internally heated, stratified, compressible convection}

\begin{abstract}
An abstract will go here eventually
\end{abstract}


\section{Introduction}
\label{sec:intro}
People who study the Sun don't understand what convective velocities are doing.
They're either way smaller than simulations and mixing length theory predict
\citep{hanasoge&all2012}, or they are roughly aligned with what we would
expect \citep{greer&all2015L}.  Understanding how convection is driven in
stellar interiors is important in constructing proper models of stellar evolution
and structure.  Thus, this convective conundrum must be figured out.

Recently, \cite{brandenburg2016} extended stellar mixing length theory to include
an additional flux term which does not depend on the local entropy gradient but
rather parameterizes the nonlocal flux carrying of convective downdrafts.  This
Deardorff flux, if sufficiently strong, could be the bulk carrier of convective
flux through an adiabatic lower convection zone, and then large ``giant cells'' won't be
driven there \cite{lord&all2014}.  Recently, \cite{kapyla&all2017} studied penetrative
convection in simulations with realistic opacities.  In these simulations, they reported
``Deardorff zones,'' or portions of the convective domain in which enthalpy flux points
upwards but the entropy gradient is positively and is nominally stable to convection.
They show that, for a specific simulation, classical penetrative convection,
such as that studied by \cite{hurlburt&all1986}, does not physically capture the
same mechanisms.  They conclude that a realistic, Kramers-like opacity
is required to study Deardorff zones.

Studies of internally heated boussinesq convection show that for the proper
boundary conditions, stable layers can be achieved \citep{goluskin&vanderpoel2016}.
Here we show that, using simple principles studies in well-understood Boussinesq
convection, stable layers are simple to achieve in internally heated atmospheres
with a simple (constant) opacity profile.  In these atmospheres, we see a reduction
of power at the surface of the atmosphere due to the stably stratified
Deardorff zone in the lower convective domain.  Deardorff zones naturally arise
in these systems, and the extent of the Deardorff zone can be understood from
the initial conditions of the atmosphere.


\begin{enumerate}
\item Importance of internal heating in natural convective systems
\item Shape of internal heating in something like Sun
\item Studies in Rayleigh-Benard
\item Goals of this paper:
\begin{enumerate}
\item Show how to set up internally heated polytropes, with stable layers (or without)
\item Show how the resultant stratification depends on combo of lower flux + internal heating profile choices
\item Show how stable layer affects surface power spectrum
\end{enumerate}
\end{enumerate}

\section{Experiment} 
We study direct numerical simulations of an ideal gas whose equation of
state is $P = \rho T$ and whose adiabatic index is $\gamma = 5/3$
by evolving the fully compressible Navier-Stokes equations, 
\label{sec:experiment}
\begin{align}
&\begin{aligned}
&\frac{\partial \ln\rho}{\partial t} + \grad\cdot\bm{u} 
    = -\bm{u}\cdot\grad\ln\rho,
	\label{eqn:continuity_eqn}
\end{aligned}\\
&\begin{aligned}
\frac{\partial\bm{u}}{\partial t} + \bm{u}\cdot\grad\bm{u} =
- \grad T - T \grad \ln \rho + \bm{g} - \grad\cdot\stressT,
\label{eqn:momentum_eqn}
\end{aligned}\\
&\begin{aligned}
\frac{\partial T}{\partial t} + \bm{u}\cdot\grad T
+ (\gamma-1)T\grad\cdot\bm{u} + \frac{1}{\rho c_V}
\grad\cdot\left(-\kappa\grad T\right) = - (\stressT\cdot\grad)\cdot\bm{u} + \kappa H,
	\label{eqn:energy_eqn}
\end{aligned}
\end{align}
with the viscous stress tensor given by
\begin{equation}
\Pi_{ij} \equiv -\mu\left(\frac{\partial u_i}{\partial x_j} + 
\frac{\partial u_j}{\partial x_i} - \frac{2}{3}\delta_{ij}\grad\cdot\bm{u}\right),
	\label{eqn:stress_tensor}
\end{equation}
where $\delta_{ij}$ is the Kronecker delta function.  In Eq. \ref{eqn:energy_eqn}, $H$
specifies the magnitude of internal heating, and in this study we study $H$ which is
constant in space and time.  We assume here that $\kappa$ and $\mu$, the thermal
conductivity and dynamic viscosity, are constant in space and time.

The initial atmosphere is constructed under the assumptions of hydrostatic equlibrium
and thermal equilibrium in the presence of constant gravity, $g = (\gamma - 1)^{-1} + 1 - \epsilon$,
where $\epsilon$ is a control parameter which sets the superadiabaticity and is similar to
the superadiabatic excess in polytropic atmospheres \cite{anders&brown2017, graham1975}.
An atmosphere which satisfies these initial conditions takes the form
\begin{equation}
T_0(z) = -\frac{H}{2} z^2 + (H L_z - 1) z + \left(1 - \frac{H}{2}L_z^2 + L_z\right),
\qquad P_0(z) = \left(\frac{\xi + \grad T_0}{\xi - \grad T_0} \cdot\frac{\xi + 1}{\xi - 1}\right)^{g/\xi},
\end{equation}
where $L_z$ is the depth of the atmosphere, $\xi \equiv \sqrt{1 + 2H}$, 
$\grad T_0 = \partial_z T_0(z) = -H z + (H_Lz - 1)$,
and the density profile is $\rho_0(z) = P_0(z)/T_0(z)$.

Stratified systems evolve towards a characteristic adiabatic profile.  An adiabatically
stratified atmosphere composed of an ideal gas in hydrostatic equilibrium has a
temperature gradient specified by the gravity, $\grad T_{ad} = -\bm{g} / c_P$, where
$c_P = \gamma/(\gamma-1)$.  In these internally heated systems,
\begin{equation}
\grad T_0 - \grad T_{ad} = H(L_z - z) - \frac{\epsilon}{c_P},
\end{equation}
and there is a special point in the initial atmosphere, $z_{\text{cross}} \equiv L_z - \epsilon / H c_P$,
at which the temperature gradient is exactly the adiabatic temperature gradient.  Above that point,
the temperature gradient is superadiabatic and unstable to convection.  Below that point,
the temperature gradient is subadiabatic.  Thus, the depth of the region that is convectively
unstable is $d_{\text{conv}} = L_z - z_{\text{cross}} = \epsilon / H c_P$.  From this,
we retrieve the magnitude of the internal heating term,
\begin{equation}
H \equiv \frac{\epsilon}{d_{\text{conv}} c_P}.
\end{equation}

If $L_z <= d_{\text{conv}}$, the whole atmosphere is unstable or marginally stable.
If $L_z > d_{\text{conv}}$, there is a stable radiative zone beneath the convective
zone.  We specify the depth of this radiative zone through a new parameter,
$r \equiv L_z/d_{\text{conv}} - 1$.  We specify the depth of the convective zone by
specifying the number of density scale heights, $n_\rho$, it spans.  To achieve this we
use an iterative, root finding algorithm find when
$f(L_z) = \rho_0(z_{\text{cross}})/\rho_0(L_z) -  e^{n_\rho}$ is zero.

Diffusivities in the system are specified by choosing a value of the Rayleigh
Number and Prandtl number.  The thermal diffusivity, $\chi = \kappa / \rho$ and
viscous diffusivity, $\nu = \mu / \rho$ are constrained by
\begin{equation}
\text{Ra}(z) = \frac{g d_{\text{conv}}^4 \bigg|\grad s / c_P\bigg|}{\nu\chi} = 
\frac{g d_{\text{conv}}^4}{\kappa\mu}\bigg|\frac{\grad s}{c_P}(z)\bigg|\rho_0^2(z),\qquad
\text{Pr} = \frac{\nu}{\chi},
\end{equation}
where 
\begin{equation}
\frac{\grad s}{c_P} = \frac{1}{\gamma} \grad\ln T - \frac{\gamma-1}{\gamma} \grad \ln \rho.
\end{equation}
We specify the value of Ra at the first moment of the $\grad T - \grad T_{ad} = T \grad s / c_P$,
\begin{equation}
L_{sm1} = \frac{\int_{z_{\text{cross}}}^{L_z} z T\grad s dz}{\int_{z_{\text{cross}}}^{L_z} T\grad s dz}.
\end{equation}
In the limit of classic, polytropic atmospheres, this reduces to the midplane of the atmosphere,
which is a commonly chosen location to specify the value of Ra \citep{hurlburt&all1984}.  We choose this location as it
minimizes the variation of the critical value of Ra as other parameters ($n_\rho$, $r$, $\epsilon$)
change.

\subsection{Stability}
We decompose thermodynamic variables such that $\ln\rho = (\ln\rho)_0 + (\ln\rho)_1$
and $T = T_0 + T_1$.  We assume that the background terms are constant with respect to
time, and this allows us to subtract out the background thermal equilibrium and
hydrostatic equilibrium.  The \emph{linearized} equations of motion are then
\begin{equation}
\begin{split}
\frac{\partial (\ln\rho)_1}{\partial t} + \grad\cdot\bm{u} + \bm{u}\cdot\grad(\ln\rho)_0 = 0 \\
\frac{\partial \bm{u}}{\partial t} + \grad T_1 + T_1 \grad(\ln\rho)_0 + T_0 \grad(\ln\rho)_1
+ \grad \cdot\stressT = 0 \\
\frac{\partial T_1}{\partial t} + \bm{u}\cdot\grad T_0 + (\gamma-1)T_0\grad\cdot\bm{u} - \kappa e^{-(\ln\rho)_0}
\grad^2 T_1 = 0
\end{split}
\end{equation}
We assume that all fluctuations $\{ T_1, (\ln\rho)_1, \bm{u} \} = f(z) g(x, y) e^{i\omega t}$, and we
use Dedalus to solve eigenvalue problems to determine when $\omega = 0$.  From this, we find
Fig. \ref{fig:onset_curves}.

\begin{figure}
\centering
\includegraphics[width=\textwidth]{./figs/onset_figure.png}
\caption{The value of the critical Rayleigh number and normalized critical wavenumber
as the control parameters of the problem are varied. \label{fig:onset_curves}}
\end{figure}


\section{Results \& Discussion}
\label{sec:results}

\bibliography{./biblio.bib}
\end{document}

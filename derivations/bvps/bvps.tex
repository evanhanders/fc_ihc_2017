%\documentclass[iop]{emulateapj}
\documentclass[aps, pre, onecolumn, nofootinbib, notitlepage, groupedaddress, amsfonts, amssymb, amsmath, longbibliography]{revtex4-1}
\usepackage{graphicx}
\usepackage{hyperref}
\usepackage{xcolor}
\hypersetup{
    colorlinks,
    linkcolor={red!50!black},
    citecolor={blue!50!black},
    urlcolor={blue!80!black}
}
\usepackage{bm}
\usepackage{natbib}
\usepackage{longtable}
\LTcapwidth=0.87\textwidth

\newcommand{\Div}[1]{\ensuremath{\nabla\cdot\left( #1\right)}}
\newcommand{\angles}[1]{\ensuremath{\left\langle #1 \right\rangle}}
\newcommand{\grad}{\ensuremath{\nabla}}
\newcommand{\RB}{Rayleigh-B\'{e}nard }
\newcommand{\stressT}{\ensuremath{\bm{\bar{\bar{\Pi}}}}}
\newcommand{\lilstressT}{\ensuremath{\bm{\bar{\bar{\sigma}}}}}
\newcommand{\nrho}{\ensuremath{n_{\rho}}}
\newcommand{\approptoinn}[2]{\mathrel{\vcenter{
	\offinterlineskip\halign{\hfil$##$\cr
	#1\propto\cr\noalign{\kern2pt}#1\sim\cr\noalign{\kern-2pt}}}}}

\newcommand{\appropto}{\mathpalette\approptoinn\relax}

\newcommand\mnras{{MNRAS}}%

\begin{document}
\author{Evan H. Anders}
\affiliation{Dept. Astrophysical \& Planetary Sciences, University of Colorado -- Boulder, Boulder, CO 80309, USA}
\affiliation{Laboratory for Atmospheric and Space Physics, Boulder, CO 80303, USA}
\author{Benjamin P. Brown}
\affiliation{Dept. Astrophysical \& Planetary Sciences, University of Colorado -- Boulder, Boulder, CO 80309, USA}
\affiliation{Laboratory for Atmospheric and Space Physics, Boulder, CO 80303, USA}
\title{BVPs to assist in convergence of IH atmospheres}

\begin{abstract}
An abstract will go here eventually
\end{abstract}
\maketitle


\section{Stellar structure models}
So in the spirit of Steve's stellar structures class (and from his notes found online,
\url{http://lasp.colorado.edu/~cranmer/ASTR_5700_2016/index.html}) I am going to
draw inspiration from stellar structure models to solve BVPs which make the thermal state of my
solutions converge more rapidly.

Stellar structure models essentially have five equations:
\begin{equation}
\begin{split}
\frac{d M_r}{dr} &= 4\pi r^2 \rho \qquad \text{(mass conservation)} \\
\frac{d P}{dr} &= -\frac{G M_r}{r^2}\rho \qquad \text{(Hydrostatic balance)} \\
\frac{d L_r}{dr} &= 4\pi r^2 \rho \epsilon \qquad \text{(Conservation of energy)} \\
\frac{d T}{dr} &= \begin{cases}
\left(\frac{dT}{dr}\right)_{\text{rad}} & \text{, if convectively stable}\\
\left(\frac{dT}{dr}\right)_{\text{ad}} - \Delta\grad T & \text{, if convectively unstable}\\
\end{cases} \qquad \text{(Basically where all of the model-dependent stuff comes in)} \\
P &= P(\rho, T, \mu)  \qquad \text{(equation of state)}
\end{split}
\end{equation}
where $\epsilon$ is the energy generation rate (erg / g / s), and $\mu$ is the
mean atomic weight, or something of the sort.

Basically, you have to solve a boundary value problem in order to find out more about the
problem.  In general, in stellar structure models, there are technically six variables,
\begin{enumerate}
\item Position: $r = [0, R_*]$
\item Mass: $M_r = [0, M_*]$
\item Mass density: $\rho = [\rho_c, \rho_{photo}]$
\item Pressure: $P = [P_c, P_{photo}]$
\item Temperature: $T = [T_c, T_{eff}]$
\item Luminosity: $L_r = [0, L_*]$
\end{enumerate}
Generally, in stellar structure models, we're interested in 6 things:
$R_*, M_*, \rho_c, P_c, T_c, L_*$.  That's six variables for five equations,
so usually $M_*$ is specified and then the rest are determined based on a boundary value problem.

...that's basically what we want to do in our problems.

\section{Our Equations}
We don't have a spherical star, we have a cartesian box.  So all of the $4\pi r^2$ area
elements turn into $xy$.  When we take a horizontal average over our simulaton, $x$ and $y$
drop out of the problem completely.  So for the 1D, $z$-direction boundary value problem that
we actually want to solve, our equations are
\begin{equation}
\begin{split}
\frac{d M}{dz} &= \rho \\
\frac{d P}{dz} &= -\rho g \hat{z} - \rho \bm{u}\cdot\grad\bm{u} - \Div{\stressT} \\
\frac{d (\text{Fluxes})}{d z} &= \kappa \text{(IH)} \\
T_{z} &= \frac{d T}{d z} \\
P = \rho T,
\end{split}
\end{equation}
where the second and third equations are the real kickers here.  In fact, the third equation is a combination
of the 3rd and 4th equation from a stellar structure model: it is conservation of energy, and it includes
all of the assumptions that we're putting into the model.

In a steady state, potential energy flux is basically zero (because the atmosphere is no longer settling).
At least, this is true for the low mach number case.  Thus, when we consider all of the fluxes in our problem,
we only need to consider enthalpy flux, KE flux, viscous flux, and conductive flux.  Thus, the third equation
above has the form
\begin{equation}
\frac{d}{dz}\left(-\kappa T_z + \rho w \left[\frac{|\bm{u}|^2}{2} + 
\left(c_V T + \frac{P}{\rho}\right)\right] + \bm{u}\cdot\stressT  \right) = \kappa \text{(IH)},
\end{equation}
where the terms are, in order, the conductive flux, the KE flux, the enthalpy flux, and the viscous flux.  The
RHS has the internal heating.

Technically, I'm dealing with a 1D problem which knows horizontal averages of an evolved 2- or 3- dimensional
convective solution.  I will use $\overline{A}$ to represent the horizontal- and time- average of a quantity
from the IVP.  Using this notation, my full set of equations to solve is
\begin{equation}
\begin{split}
\frac{d M}{dz} &= \rho \\
\overline{\frac{d P}{dz}} &= \overline{-\rho g \hat{z} - \rho \bm{u}\cdot\grad\bm{u} - \Div{\stressT}} \\
\overline{\frac{d}{dz}\left(-\kappa T_z + \rho w\left[\frac{|\bm{u}|^2}{2} + 
\left(c_V T + \frac{P}{\rho}\right)\right] + \bm{u}\cdot\stressT  \right)} &= \overline{\kappa \text{(IH)}}, \\
T_z &= \frac{dT}{dz} \\
P &= \rho T,
\end{split}
\end{equation}
Here, $\overline{A} = \iiint A \,dx \,dy \,dt / (L_y\cdot L_x\cdot T)$ is the time- and horizontally-
averaged profile of the variable $A$, where $T$ is the time interval over which an average is taken.
Fortunately, $d/dz$ commutes with this operation, so we can just put it inside of any z-derivatives
and we should be fine.  Let's break things up really carefully.

\subsection{Momentum equation}
There's really two parts of the momentum equation: the part that we normally think of
as hydrostatic balance, and the velocity parts.  Let's look at the former, first. We have
\begin{equation}
\frac{d\bar{P}}{dz} + \overline{\rho g} = \frac{d}{dz}\left( \overline{\rho T}\right) + \overline{\rho g}
= \overline{T d_z \rho + \rho d_z T} + \overline{\rho g}
\end{equation}
Temperature and density are broken up such that
\begin{equation}
T \equiv T_0 + T_{IVP} + T_1; \qquad \rho \equiv \rho_0 + \rho_{IVP} + \rho_1.
\end{equation}
Here, $T_{IVP}$ is just the temperature fluctuations from the IVP, and $\rho_{IVP} = \rho_0 (e^{\ln\rho_1} - 1)$
of the IVP.  The ``subscript 1'' variables here are the fluctuations that will be solved for in the BVP.  With
that in mind, my previous HS balance equation is:
\begin{equation}
\overline{T d_z \rho + \rho d_z T} + \overline{\rho g} =
\overline{(T_0 + T_{IVP} + T_1) d_z(\rho_0 + \rho_{IVP} + \rho_1) + (\rho_0 + \rho_{IVP} + \rho_1)d_z(T \equiv T_0 + T_{IVP} + T_1)
+ (\rho_0 + \rho_{IVP} + \rho_1)g}
\end{equation}
And at this point, we're too long for one line, so
\begin{equation}
\begin{split}
&\overline{(T_0 + T_{IVP} + T_1) d_z(\rho_0 + \rho_{IVP} + \rho_1)} = 
		\overline{(T_0 + T_{IVP})d_z(\rho_0 + \rho_{IVP})} + \overline{(T_0 + T_{IVP})}d_z(\rho_1)
		+ T_1\overline{d_z(\rho_0 + \rho_{IVP})} + T_1 d_z(\rho_1) \\
&\overline{(\rho_0 + \rho_{IVP} + \rho_1)d_z(T_0 + T_{IVP} + T_1)} =
		\overline{(\rho_0 + \rho_{IVP})d_z(T_0 + T_{IVP})} + \overline{(\rho_0 + \rho_{IVP})d_z(T_1)}
		+ \rho_1\overline{d_z(T_0 + T_{IVP})} + \rho_1 d_z(T_1) \\
&\overline{(\rho_0 + \rho_{IVP} + \rho_1)g} = \overline{(\rho_0 + \rho_{IVP})}g + \rho_1 g,
\end{split}
\end{equation}
where in the RHS expressions I have taken $\rho_1$ and $T_1$ out of the bars, because they are
definitionally vertical profiles with no time- or horizontal- variance.  All of the yucky terms which have
bars over them are terms that I should \emph{directly find a time- and horizontal- average of} if I want
to have precisely the right BVP.  

The rest of the momentum equation is fairly easy:
\begin{equation}
\overline{\rho\bm{u}\cdot\grad w - \Div{\stressT}_z} =
\overline{(\rho_0 + \rho_{IVP})\bm{u}\cdot\grad w} + \rho_1\overline{\bm{u}\cdot\grad w}
- \overline{\Div{\stressT}_z}.
\end{equation}
Here, the stress tensor term depends only on velocity and $\kappa$, and since I am solving systems with constant
$\kappa$, I don't need to worry about breaking it up any more.  That means that I need the
following profiles going into my BVP for the momentum equation:
\begin{enumerate}
\item \texttt{T0\_full =}		 $\overline{(T_0 + T_{IVP})}$
\item \texttt{T0\_z\_full =}		 $\overline{d_z(T_0 + T_{IVP})}$
\item \texttt{rho0\_full =}		 $\overline{(\rho_0 + \rho_{IVP})}$
\item \texttt{rho0\_z\_full =}	 $\overline{d_z(\rho_0 + \rho_{IVP})}$
\item \texttt{T\_grad\_rho =}		 $\overline{(T_0 + T_{IVP})d_z(\rho_0 + \rho_{IVP})}$
\item \texttt{rho\_grad\_T =}      $\overline{(\rho_0 + \rho_{IVP})d_z(T_0 + T_{IVP})}$
\item \texttt{rho\_uDotGradw =}   $\overline{(\rho_0 + \rho_{IVP})\bm{u}\cdot\grad w}$
\item \texttt{uDotGradw =}       $\overline{\bm{u}\cdot\grad w}$
\item \texttt{visc\_w =}         $\overline{\Div{\stressT}_z}$
\end{enumerate}

\subsection{Energy Equation}
So this equation is $\Div{\text{fluxes}} = \kappa IH$.  The RHS is already just a constant of
the system (or a constant profile in time if we want to add that complexity later).  The LHS
needs some love.  Let's examine each flux individually.
\begin{equation}
\overline{\text{conductive flux}} = \overline{- \kappa \grad(T_0 + T_{IVP} + T_1)}
= -\kappa (\overline{\grad(T_0 + T_{IVP})} + \grad T_1),
\end{equation}
so...yeah, that one's super simple for the constant kappa case.
\begin{equation}
\overline{\text{viscous flux}} = \overline{(\bm{u}\cdot\stressT)_w},
\end{equation}
and I might be off by a negative sign here (should check, but I think I'm not).  This is just a
function of $\kappa$ and $\bm{u}$, so once again...here, we have it easy.
\begin{equation}
\overline{\text{KE flux}} = \overline{\rho w (\bm{u})^2 / 2}
= \overline{(\rho_0 + \rho_{IVP}) w (\bm{u})^2 / 2} + \rho_1 \overline{w (\bm{u})^2/2},
\end{equation}
which is slightly more rough, but not bad.  Then there's
\begin{equation}
\overline{\text{enthalpy flux}} = \overline{\rho w T (C_v + 1)}
= (C_v + 1)\left(\overline{(\rho_0 + \rho_{IVP})(T_0 + T_{IVP}) w}
+ \overline{(\rho_0 + \rho_{IVP}) w} T_1 + \rho_1\overline{w (T_0 + T_{IVP})}
+ \rho_1 T_1 \overline{w}
\right)
\end{equation}
and we're ignoring PE flux, by choice.

So in the end, this energy equation requires the following \emph{new} things that we didn't have
from the momentum equation
\begin{enumerate}
\item \texttt{visc\_flux =} 		$\overline{(\bm{u}\cdot\stressT)_w}$
\item \texttt{KE\_flux\_IVP =}		$\overline{(\rho_0 + \rho_{IVP}) w (\bm{u})^2 / 2}$
\item \texttt{w\_vel\_squared =}	$\overline{w (\bm{u})^2 / 2}$
\item \texttt{Enth\_flux\_IVP =}	$\overline{(C_v + 1)(\rho_0 + \rho_{IVP})(T_0 + T_{IVP}) w}$
\item \texttt{rho\_w\_IVP =}		$\overline{(\rho_0 + \rho_{IVP})w}$
\item \texttt{T\_w\_IVP =}		    $\overline{(T_0 + T_{IVP}) w}$
\item \texttt{w\_IVP =}				$\overline{w}$
\end{enumerate}




\subsection{Implementation in Dedalus}
With the substitutions from the above sections, my four equations that I implement in dedalus are
\begin{enumerate}
\item \texttt{dz(M1) - rho1 = 0}
\item \texttt{dz(T1) - T1\_z = 0}
\item \texttt{dz(}$-\kappa$\texttt{T1\_z + rho1 * (w\_vel\_squared + T\_w\_IVP*(Cv + 1)) + T1 * rho\_w\_IVP* (Cv + 1)) =} \\
	  \texttt{-dz(}$-\kappa$\texttt{T0\_z\_full + visc\_flux + KE\_flux\_IVP + Enth\_flux\_IVP + (Cv+1)*rho1*T1*w\_IVP) +} $\kappa$\texttt{(IH)}
\item \texttt{T0\_full*dz(rho1) + T1*rho0\_z\_full + rho0\_full*dz(T1) + rho1 * T0\_z\_full + rho1 g + rho1 * uDotGradw} \\
	  \texttt{= -T\_grad\_rho - T1*dz(rho1) - rho\_grad\_T - rho1*dz(T1) - rho0\_full * g - rho\_uDotGradw - visc\_w }
\end{enumerate}
The boundary conditions of this system are then
\begin{enumerate}
\item \texttt{left(M1) = 0}
\item \texttt{right(M1) = 0}
\item \texttt{left(T1\_z) = 0}
\item \texttt{right(T1) = 0}
\end{enumerate}
The last two of these conditions are just the standard thermal boundary conditions used in these
simulations.  The first two conditions ensure that no mass is added to the system.  The structure of
the dz(fluxes) = IH equation ensures that flux equilibrium is met throughout the atmosphere, and the
dz(P) equation ensures that there are no $m = 0$ pressure imbalances in the atmosphere.

Sweetness.


\bibliography{../biblio.bib}
\end{document}

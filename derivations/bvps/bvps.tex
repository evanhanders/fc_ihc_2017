%\documentclass[iop]{emulateapj}
\documentclass[aps, pre, onecolumn, nofootinbib, notitlepage, groupedaddress, amsfonts, amssymb, amsmath, longbibliography]{revtex4-1}
\usepackage{graphicx}
\usepackage{hyperref}
\usepackage{xcolor}
\hypersetup{
    colorlinks,
    linkcolor={red!50!black},
    citecolor={blue!50!black},
    urlcolor={blue!80!black}
}
\usepackage{bm}
\usepackage{natbib}
\usepackage{longtable}
\LTcapwidth=0.87\textwidth

\newcommand{\Div}[1]{\ensuremath{\nabla\cdot\left( #1\right)}}
\newcommand{\angles}[1]{\ensuremath{\left\langle #1 \right\rangle}}
\newcommand{\grad}{\ensuremath{\nabla}}
\newcommand{\RB}{Rayleigh-B\'{e}nard }
\newcommand{\stressT}{\ensuremath{\bm{\bar{\bar{\Pi}}}}}
\newcommand{\lilstressT}{\ensuremath{\bm{\bar{\bar{\sigma}}}}}
\newcommand{\nrho}{\ensuremath{n_{\rho}}}
\newcommand{\approptoinn}[2]{\mathrel{\vcenter{
	\offinterlineskip\halign{\hfil$##$\cr
	#1\propto\cr\noalign{\kern2pt}#1\sim\cr\noalign{\kern-2pt}}}}}

\newcommand{\appropto}{\mathpalette\approptoinn\relax}

\newcommand\mnras{{MNRAS}}%

\begin{document}
\title{BVPs to assist in convergence of IH atmospheres}
\maketitle

\section{The FC Boundary value equations}
These equations are motivated by stellar structure models (see \ref{sec:stellar_models}).
One important thing that stellar structure models get right is that they know how to
conserve mass in a 1D BVP, so that aspect of these equations is basically stolen right
from them.  We're dealing with systems where we \emph{know what the flux profile should
look in the equilibrated state} a priori, so we can tap into that knowledge.  We use
a modified hydrostatic equlibrium which accounts for advective and viscous pressure
support, and we also don't use a MLT model for how the temperature gradient evolves.
Rather, we solve for that using the full fluxes.

So, that being said, our system of five equations (to parallel the system of 
five equations you have to solve in structure models) becomes
\begin{equation}
\begin{split}
\angles{\frac{d M}{dz}} &= \angles{\rho} \\
\angles{\frac{d P}{dz}} &= \angles{-\rho g \hat{z} - \rho (\bm{u}\cdot\grad)w - \Div{\stressT}_z} \\
\angles{\frac{d (\text{Fluxes}_z)}{d z}} &= \angles{\kappa \text{(IH)}} \\
\angles{T_{z}} &= \angles{\frac{d T}{d z}} \\
\angles{P} &= \angles{\rho T},
\end{split}
\end{equation}
Here, $\angles{A} = \iiint A \,dx\,dy\,dt / (L_x\cdot L_y \cdot T)$ is the time- and horizontal-
average of the quantity $A$, where $T$ is the quantity of time over which we're time-averaging,
and ``angles'' should commute with $z$-derivatives.
All subscript $z$ quantities mean that we are only examining the z-component of the full vector.
In these equations, the viscous stress tensor is defined as
\begin{equation}
\Pi_{ij} \equiv -\mu\left(\frac{\partial u_i}{\partial x_j} + \frac{\partial u_j}{\partial x_i} -
- \frac{2}{3}\delta_{ij}\Div{\bm{u}}\right).
\label{eqn:stress_tensor}
\end{equation}

So the important thing is to come up with a proper model for which fluxes carry the convection.
In general, we don't want to neglect anything, but we also know that \emph{potential energy
flux is usually associated with a transient state} and restratification of the atmosphere.
Thus, I impose that
\begin{equation}
\angles{\rho w \phi} = 0,
\end{equation}
and we will ignore that flux from here on out.  Under that assumption, we still have to
worry about the conductive flux, enthalpy flux, viscous flux, and KE flux.  Or, in
equation form,
\begin{equation}
\angles{\text{Fluxes}_z} = \angles{\rho w \left(c_V T + \frac{P}{\rho}\right)}
+ \angles{\rho w \frac{|\bm{u}|^2}{2}} + \angles{-\kappa \frac{d T}{dz}}
+ \angles{(\bm{u}\cdot\stressT)_z}
\end{equation}
...and that's the system!  It's important for me to note at this point that the
variables that I solve for in the BVPs are $T$, $dT/dz$, and $\rho$ (and $M$).
In other words, I solve for the \emph{thermodynamic structure} of the atmosphere
that is in flux equilibrium and is \emph{no longer evolving} (no PE flux) given the
current velocity field fed in from an IVP.  

Once we plug the EOS into the other equations,
we're left with four equations in our systems.  The boundary conditions I will specify are:
\begin{enumerate}
\item left(M) = 0, right(M) = integrated mass of initial atmosphere.  The BVP is
not allowed to add mass to the problem.  This is two boundary conditions.
\item right($T_1$) = 0, fixed temperature top boundary, just like in my IVP.
\item left($dz(T_1)$) = 0, fixed flux bottom boundary, which specifies exactly the shape
that the full flux profile will take (and the full amount of flux at any point in the
system).
\end{enumerate}

It's also important for me to say that these equations work well for constant values
of $\kappa$ and $\mu$ (in both time and space).  I haven't thought about the problem
for variable $\kappa$/$\mu$, and while it will still work, it will be more complicated.


\subsection{Implementing the Momentum equation}
There's really two parts of the momentum equation: the part that we normally think of
as hydrostatic balance, and the velocity parts.  Let's look at the former, first. We have
\begin{equation}
\angles{\frac{dP}{dz}} + \angles{\rho g} = \angles{\frac{d}{dz}\left( \rho T\right) } + \angles{\rho g}
= \angles{T d_z \rho + \rho d_z T} + \angles{\rho g}
\end{equation}
Temperature and density are broken up such that
\begin{equation}
T \equiv T_0 + T_{IVP} + T_1; \qquad \rho \equiv \rho_0 + \rho_{IVP} + \rho_1.
\end{equation}
Here, $T_{IVP}$ is just the temperature fluctuations from the IVP, and $\rho_{IVP} = \rho_0 (e^{\ln\rho_1} - 1)$
of the IVP.  The ``subscript 1'' variables here are the fluctuations that will be solved for in the BVP.  With
that in mind, my previous HS balance equation is:
\begin{equation}
\angles{T d_z \rho + \rho d_z T} + \angles{\rho g} =
\angles{(T_0 + T_{IVP} + T_1) d_z(\rho_0 + \rho_{IVP} + \rho_1) + (\rho_0 + \rho_{IVP} + \rho_1)d_z(T_0 + T_{IVP} + T_1)
+ (\rho_0 + \rho_{IVP} + \rho_1)g}
\end{equation}
And at this point, we're too long for one line, so
\begin{equation}
\begin{split}
&\angles{(T_0 + T_{IVP} + T_1) d_z(\rho_0 + \rho_{IVP} + \rho_1)} = 
		\angles{(T_0 + T_{IVP})d_z(\rho_0 + \rho_{IVP})} + \angles{(T_0 + T_{IVP})}d_z(\rho_1)
		+ T_1\angles{d_z(\rho_0 + \rho_{IVP})} + T_1 d_z(\rho_1) \\
&\angles{(\rho_0 + \rho_{IVP} + \rho_1)d_z(T_0 + T_{IVP} + T_1)} =
		\angles{(\rho_0 + \rho_{IVP})d_z(T_0 + T_{IVP})} + \angles{(\rho_0 + \rho_{IVP})}d_z(T_1)
		+ \rho_1\angles{d_z(T_0 + T_{IVP})} + \rho_1 d_z(T_1) \\
&\angles{(\rho_0 + \rho_{IVP} + \rho_1)g} = \angles{(\rho_0 + \rho_{IVP})}g + \rho_1 g,
\end{split}
\end{equation}
where in the RHS expressions I have taken $\rho_1$ and $T_1$ out of the angles, because they are
definitionally vertical profiles with no time- or horizontal- variance (that's what it means
for them to be the solution to the BVP).  All of the yucky terms which have
angles around them are terms that I should \emph{directly find a time- and horizontal- average of} if I want
to have precisely the right BVP.  

The rest of the momentum equation is fairly easy:
\begin{equation}
\angles{-\rho\bm{u}\cdot\grad w - \Div{\stressT}_z} =
- \angles{(\rho_0 + \rho_{IVP})\bm{u}\cdot\grad w} - \rho_1\angles{\bm{u}\cdot\grad w}
- \angles{\Div{\stressT}_z}.
\end{equation}
Here, the stress tensor term depends only on velocity and $\kappa$, and since I am solving systems with constant
$\kappa$, I don't need to worry about breaking it up any more.  That means that I need the
following profiles going into my BVP for the momentum equation:
\begin{enumerate}
\item \texttt{T0\_full =}		 $\angles{(T_0 + T_{IVP})}$
\item \texttt{T0\_z\_full =}		 $\angles{d_z(T_0 + T_{IVP})}$ (honestly, this should just be dz(above) )
\item \texttt{rho0\_full =}		 $\angles{(\rho_0 + \rho_{IVP})}$
\item \texttt{rho0\_z\_full =}	 $\angles{d_z(\rho_0 + \rho_{IVP})}$ (honestly, this should just be dz(above) )
\item \texttt{T\_grad\_rho =}		 $\angles{(T_0 + T_{IVP})d_z(\rho_0 + \rho_{IVP})}$
\item \texttt{rho\_grad\_T =}      $\angles{(\rho_0 + \rho_{IVP})d_z(T_0 + T_{IVP})}$
\item \texttt{rho\_uDotGradw =}   $\angles{(\rho_0 + \rho_{IVP})\bm{u}\cdot\grad w}$
\item \texttt{uDotGradw =}       $\angles{\bm{u}\cdot\grad w}$
\item \texttt{visc\_w =}         $\angles{\Div{\stressT}_z}$
\end{enumerate}

\subsection{Energy Equation}
So this equation is $\Div{\text{fluxes}} = \kappa IH$.  The RHS is already just a constant of
the system (or a constant profile in time if we want to add that complexity later).  The LHS
needs some love.  Let's examine each flux individually.
\begin{equation}
\angles{\text{conductive flux}} = \angles{- \kappa d_z(T_0 + T_{IVP} + T_1)}
= -\kappa (\angles{d_z(T_0 + T_{IVP})} + d_z T_1),
\end{equation}
so...yeah, that one's super simple for the constant kappa case.
\begin{equation}
\angles{\text{viscous flux}} = \angles{(\bm{u}\cdot\stressT)_z},
\end{equation}
This is just a function of $\kappa$ and $\bm{u}$, so once again...here, we have it easy.
\begin{equation}
\angles{\text{KE flux}} = \angles{\rho w (\bm{u})^2 / 2}
= \angles{(\rho_0 + \rho_{IVP}) w (\bm{u})^2 / 2} + \rho_1 \angles{w (\bm{u})^2/2},
\end{equation}
which is slightly more rough, but not bad.  Then there's
\begin{equation}
\angles{\text{enthalpy flux}} = \angles{\rho w T (C_v + 1)}
= (C_v + 1)\left\{\angles{(\rho_0 + \rho_{IVP})(T_0 + T_{IVP}) w}
+ \angles{(\rho_0 + \rho_{IVP}) w} T_1 + \rho_1\angles{w (T_0 + T_{IVP})}
+ \rho_1 T_1 \angles{w}
\right\}
\end{equation}
and we're ignoring PE flux, by choice.

So in the end, this energy equation requires the following \emph{new} things that we didn't have
from the momentum equation
\begin{enumerate}
\item \texttt{visc\_flux =} 		$\angles{(\bm{u}\cdot\stressT)_w}$
\item \texttt{KE\_flux\_IVP =}		$\angles{(\rho_0 + \rho_{IVP}) w (\bm{u})^2 / 2}$
\item \texttt{w\_vel\_squared =}	$\angles{w (\bm{u})^2 / 2}$
\item \texttt{Enth\_flux\_IVP =}	$\angles{(C_v + 1)(\rho_0 + \rho_{IVP})(T_0 + T_{IVP}) w}$
\item \texttt{rho\_w\_IVP =}		$\angles{(\rho_0 + \rho_{IVP})w}$
\item \texttt{T\_w\_IVP =}		    $\angles{(T_0 + T_{IVP}) w}$
\item \texttt{w\_IVP =}				$\angles{w}$
\end{enumerate}




\subsection{Implementation in Dedalus}
With the substitutions from the above sections, my four equations that I implement in dedalus are
\begin{enumerate}
\item \texttt{dz(M1) - rho1 = 0}
\item \texttt{dz(T1) - T1\_z = 0}
\item \texttt{dz(}$-\kappa$\texttt{T1\_z + rho1 * (w\_vel\_squared + T\_w\_IVP * (Cv + 1)) + T1 * rho\_w\_IVP * (Cv + 1)) =} \\
	  \texttt{-dz(}$-\kappa$\texttt{T0\_z\_full + visc\_flux + KE\_flux\_IVP + Enth\_flux\_IVP + (Cv+1) * rho1 * T1 * w\_IVP) +} $\kappa$\texttt{(IH)}
\item \texttt{T0\_full*dz(rho1) + T1*rho0\_z\_full + rho0\_full*dz(T1) + rho1 * T0\_z\_full + rho1*g + rho1 * uDotGradw} \\
	  \texttt{= -T\_grad\_rho - T1*dz(rho1) - rho\_grad\_T - rho1*dz(T1) - rho0\_full * g - rho\_uDotGradw - visc\_w }
\end{enumerate}
The boundary conditions of this system are then
\begin{enumerate}
\item \texttt{left(M1) = 0}
\item \texttt{right(M1) = 0}
\item \texttt{left(T1\_z) = 0}
\item \texttt{right(T1) = 0}
\end{enumerate}
The last two of these conditions are just the standard thermal boundary conditions used in these
simulations.  The first two conditions ensure that no mass is added to the system.  The structure of
the dz(fluxes) = IH equation ensures that flux equilibrium is met throughout the atmosphere, and the
dz(P) equation ensures that there are no $m = 0$ pressure imbalances in the atmosphere.

I'm having a bit of trouble finding what tolerance I should set in my BVP.  I was initially using
$10^{-3}\epsilon$, but I'm finding that that doesn't converge things well enough.  I'm now
using $10^{-10}\epsilon$, and that seems to be working better, but I haven't had the
time to thoroughly test it.

\section{Stellar structure models}
This is where my motivation for the current system of equations lie.
These equations are from Steve's stellar structures class (and from his notes found online,
\url{http://lasp.colorado.edu/~cranmer/ASTR_5700_2016/index.html}) 
Stellar structure models essentially have five equations:
\begin{equation}
\begin{split}
\frac{d M_r}{dr} &= 4\pi r^2 \rho \qquad \text{(mass conservation)} \\
\frac{d P}{dr} &= -\frac{G M_r}{r^2}\rho \qquad \text{(Hydrostatic balance)} \\
\frac{d L_r}{dr} &= 4\pi r^2 \rho \epsilon \qquad \text{(Conservation of energy)} \\
\frac{d T}{dr} &= \begin{cases}
\left(\frac{dT}{dr}\right)_{\text{rad}} & \text{, if convectively stable}\\
\left(\frac{dT}{dr}\right)_{\text{ad}} - \Delta\grad T & \text{, if convectively unstable}\\
\end{cases} \qquad \text{(Basically where all of the model-dependent stuff comes in)} \\
P &= P(\rho, T, \mu)  \qquad \text{(equation of state)}
\end{split}
\end{equation}
where $\epsilon$ is the energy generation rate (erg / g / s), and $\mu$ is the
mean atomic weight, or something of the sort.  In my above system of equations,
I'm directly implementing three of these equations (mass conservation, hydrostatic
balance, and EOS).  The other two equations (conservation of energy and the mixing
length assumptions that go into $\Delta \grad T$) are combined in my energy equation
in the following way:  I know what the ``luminosity'' of my atmosphere is because
the internal heating term is ramping up the amount of flux being carried over the
system.  This is something I know exactly ($\kappa$ IH).  Then, rather than making an assumption
about how the temperature gradient should evolve, I just assume that the temperature field is the thing
that gives me flux equilibrium in the presence of all of the other forcings.

So....yeah, it's basically this type of system, but tailored to fit simpler atmospheres.


\bibliography{../biblio.bib}
\end{document}

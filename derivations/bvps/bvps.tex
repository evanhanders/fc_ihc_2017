%\documentclass[iop]{emulateapj}
\documentclass[aps, pre, onecolumn, nofootinbib, notitlepage, groupedaddress, amsfonts, amssymb, amsmath, longbibliography]{revtex4-1}
\usepackage{graphicx}
\usepackage{hyperref}
\usepackage{xcolor}
\hypersetup{
    colorlinks,
    linkcolor={red!50!black},
    citecolor={blue!50!black},
    urlcolor={blue!80!black}
}
\usepackage{bm}
\usepackage{natbib}
\usepackage{longtable}
\LTcapwidth=0.87\textwidth

\newcommand{\Div}[1]{\ensuremath{\nabla\cdot\left( #1\right)}}
\newcommand{\angles}[1]{\ensuremath{\left\langle #1 \right\rangle}}
\newcommand{\grad}{\ensuremath{\nabla}}
\newcommand{\RB}{Rayleigh-B\'{e}nard }
\newcommand{\stressT}{\ensuremath{\bm{\bar{\bar{\Pi}}}}}
\newcommand{\lilstressT}{\ensuremath{\bm{\bar{\bar{\sigma}}}}}
\newcommand{\nrho}{\ensuremath{n_{\rho}}}
\newcommand{\approptoinn}[2]{\mathrel{\vcenter{
	\offinterlineskip\halign{\hfil$##$\cr
	#1\propto\cr\noalign{\kern2pt}#1\sim\cr\noalign{\kern-2pt}}}}}

\newcommand{\appropto}{\mathpalette\approptoinn\relax}

\newcommand\mnras{{MNRAS}}%

\begin{document}
\author{Evan H. Anders}
\affiliation{Dept. Astrophysical \& Planetary Sciences, University of Colorado -- Boulder, Boulder, CO 80309, USA}
\affiliation{Laboratory for Atmospheric and Space Physics, Boulder, CO 80303, USA}
\author{Benjamin P. Brown}
\affiliation{Dept. Astrophysical \& Planetary Sciences, University of Colorado -- Boulder, Boulder, CO 80309, USA}
\affiliation{Laboratory for Atmospheric and Space Physics, Boulder, CO 80303, USA}
\title{BVPs to assist in convergence of IH atmospheres}

\begin{abstract}
An abstract will go here eventually
\end{abstract}
\maketitle


\section{Stellar structure models}
So in the spirit of Steve's stellar structures class (and from his notes found online,
\url{http://lasp.colorado.edu/~cranmer/ASTR_5700_2016/index.html}) I am going to
draw inspiration from stellar structure models to solve BVPs which make the thermal state of my
solutions converge more rapidly.

Stellar structure models essentially have five equations:
\begin{equation}
\begin{split}
\frac{d M_r}{dr} &= 4\pi r^2 \rho \qquad \text{(mass conservation)} \\
\frac{d P}{dr} &= -\frac{G M_r}{r^2}\rho \qquad \text{(Hydrostatic balance)} \\
\frac{d L_r}{dr} &= 4\pi r^2 \rho \epsilon \qquad \text{(Conservation of energy)} \\
\frac{d T}{dr} &= \begin{cases}
\left(\frac{dT}{dr}\right)_{\text{rad}} & \text{, if convectively stable}\\
\left(\frac{dT}{dr}\right)_{\text{ad}} - \Delta\grad T & \text{, if convectively unstable}\\
\end{cases} \qquad \text{(Basically where all of the model-dependent stuff comes in)} \\
P &= P(\rho, T, \mu)  \qquad \text{(equation of state)}
\end{split}
\end{equation}
where $\epsilon$ is the energy generation rate (erg / g / s), and $\mu$ is the
mean atomic weight, or something of the sort.

Basically, you have to solve a boundary value problem in order to find out more about the
problem.  In general, in stellar structure models, there are technically six variables,
\begin{enumerate}
\item Position: $r = [0, R_*]$
\item Mass: $M_r = [0, M_*]$
\item Mass density: $\rho = [\rho_c, \rho_{photo}]$
\item Pressure: $P = [P_c, P_{photo}]$
\item Temperature: $T = [T_c, T_{eff}]$
\item Luminosity: $L_r = [0, L_*]$
\end{enumerate}
Generally, in stellar structure models, we're interested in 6 things:
$R_*, M_*, \rho_c, P_c, T_c, L_*$.  That's six variables for five equations,
so usually $M_*$ is specified and then the rest are determined based on a boundary value problem.

...that's basically what we want to do in our problems.

\section{Our Equations}
We don't have a spherical star, we have a cartesian box.  So all of the $4\pi r^2$ area
elements turn into $xy$.  When we take a horizontal average over our simulaton, $x$ and $y$
drop out of the problem completely.  So for the 1D, $z$-direction boundary value problem that
we actually want to solve, our equations are
\begin{equation}
\begin{split}
\frac{d M}{dz} &= \rho \\
\frac{d P}{dz} &= -\rho g \hat{z} - \rho \bm{u}\cdot\grad\bm{u} - \Div{\stressT} \\
\frac{d (\text{Fluxes})}{d z} &= \kappa \text{(IH)} \\
T_{z} &= \frac{d T}{d z} \\
P = \rho T,
\end{split}
\end{equation}
where the second and third equations are the real kickers here.  In fact, the third equation is a combination
of the 3rd and 4th equation from a stellar structure model: it is conservation of energy, and it includes
all of the assumptions that we're putting into the model.

In a steady state, potential energy flux is basically zero (because the atmosphere is no longer settling).
At least, this is true for the low mach number case.  Thus, when we consider all of the fluxes in our problem,
we only need to consider enthalpy flux, KE flux, viscous flux, and conductive flux.  Thus, the third equation
above has the form
\begin{equation}
\frac{d}{dz}\left(-\kappa T_z + \rho w \left[\frac{|\bm{u}|^2}{2} + 
\left(c_V T + \frac{P}{\rho}\right)\right] + \bm{u}\cdot\stressT  \right) = \kappa \text{(IH)},
\end{equation}
where the terms are, in order, the conductive flux, the KE flux, the enthalpy flux, and the viscous flux.  The
RHS has the internal heating.

Technically, I'm dealing with a 1D problem which knows horizontal averages of an evolved 2- or 3- dimensional
convective solution.  I will use $\overline{A}$ to represent the horizontal- and time- average of a quantity
from the IVP.  Using this notation, my full set of equations to solve is
\begin{equation}
\begin{split}
\frac{d M}{dz} &= \rho \\
\frac{d P}{dz} &= -\rho g \hat{z} - \rho \overline{\bm{u}\cdot\grad\bm{u}} - \overline{\Div{\stressT}} \\
\frac{d}{dz}\left(-\kappa T_z + \rho\overline{w}\left[\frac{\overline{|\bm{u}|^2}}{2} + 
\left(c_V T + \frac{P}{\rho}\right)\right] + \overline{\bm{u}\cdot\stressT}  \right) &= \kappa \text{(IH)}, \\
T_z &= \frac{dT}{dz} \\
P &= \rho T,
\end{split}
\end{equation}
All variables with overbars must be measured directly from the evolved solution.  Technically, there's some
problem here with my averaging -- I'm averaging over *part* of some of the terms, but not all of them
(because $\rho$, etc. are variables that are evolving).  I need to think about this more in the future,
but for the low $\epsilon$ case, what I'm doing seems to work fairly well.

\subsection{Implementation in Dedalus}
Anyways, I'm interested specifically in getting this to work in dedalus.  Thus far, I think we have four
important variables: $M_1$, the fluctuation in the total mass around the background;
$\rho_1$, fluctuations in the density profile; $T_1$, fluctuations in the temperature profile; and
$dT_1 / dz$, fluctuations in the temperature profile's gradients.  The background values
($M_0$, $\rho_0$, $T_0$, $dT_0/dz$) should all be set by a combination of the
initial conductive reference state, and the evolved time- and horizontally- averaged profiles from the
IVP.

The boundary conditions are:
\begin{equation}
\begin{split}
M_1(z=0) &= 0 \\
M_1(z=L_z) &= 0 \\
T_1(z=L_z) &= 0 \\
\frac{dT_1}{dz}(z=0) &= 0.
\end{split}
\end{equation}
The last two of these conditions are just the standard thermal boundary conditions used in these
simulations.  The first two conditions ensure that no mass is added to the system.  The structure of
the dz(fluxes) = IH equation ensures that flux equilibrium is met throughout the atmosphere, and the
dz(P) equation ensures that there are no $m = 0$ pressure imbalances in the atmosphere.

Sweetness.


\bibliography{../biblio.bib}
\end{document}

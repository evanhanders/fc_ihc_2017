%\documentclass[iop]{emulateapj}
\documentclass[aps, pre, onecolumn, nofootinbib, notitlepage, groupedaddress, amsfonts, amssymb, amsmath, longbibliography]{revtex4-1}
\usepackage{graphicx}
\usepackage{hyperref}
\usepackage{xcolor}
\hypersetup{
    colorlinks,
    linkcolor={red!50!black},
    citecolor={blue!50!black},
    urlcolor={blue!80!black}
}
\usepackage{bm}
\usepackage{natbib}
\usepackage{longtable}
\LTcapwidth=0.87\textwidth

\newcommand{\Div}[1]{\ensuremath{\nabla\cdot\left( #1\right)}}
\newcommand{\angles}[1]{\ensuremath{\left\langle #1 \right\rangle}}
\newcommand{\grad}{\ensuremath{\nabla}}
\newcommand{\RB}{Rayleigh-B\'{e}nard }
\newcommand{\stressT}{\ensuremath{\bm{\bar{\bar{\Pi}}}}}
\newcommand{\lilstressT}{\ensuremath{\bm{\bar{\bar{\sigma}}}}}
\newcommand{\nrho}{\ensuremath{n_{\rho}}}
\newcommand{\approptoinn}[2]{\mathrel{\vcenter{
	\offinterlineskip\halign{\hfil$##$\cr
	#1\propto\cr\noalign{\kern2pt}#1\sim\cr\noalign{\kern-2pt}}}}}

\newcommand{\appropto}{\mathpalette\approptoinn\relax}

\newcommand\mnras{{MNRAS}}%

\begin{document}
\title{FC equations with Internal Heating, non-dimensionalized}

\maketitle

\section{FC Equations}
The fully compressible Navier-Stokes equations, which we will solve, are:
\begin{equation}
\begin{split}
\frac{D \ln \rho}{D t} + \grad\cdot(\bm{u}) &= 0 \\
\rho\frac{D \bm{u}}{D t} &= -\grad P + \rho \bm{g} - \grad\cdot\stressT \\
\rho c_V \left(\frac{D T}{D t} + (\gamma - 1)T\grad\cdot\bm{u}\right) + 
\grad\cdot ( - \kappa \grad T) &= - (\stressT \cdot \grad) \cdot \bm{u} + \kappa H
\end{split}
\end{equation}
Which is to say that we've added a term to the energy equation, $\kappa H$, where
$\kappa$ and $H$ are both constants, and so this is a source term that will internally
heat the system.

\subsection{Non-dimensionalization}
In order to non-dimensionalize, we need to essentially define new variables in
terms of our old variables, where we have removed an important scale factor from
each which in some way describes a physical scale of our problem.
Thus, I define:
\begin{equation}
\begin{split}
\grad &= \frac{1}{\bar{L}}\grad^* \\
t &= \bar{t} t^* \\
T &= \bar{T} T^* \\
u &= \bar{u} u^* = \frac{\bar{L}}{\bar{t}} u^*
\end{split}
\end{equation}
Here, variables with over-bars are the dimension-full, characteristic scales of my problem,
and starred variables are my non-dimensional, time-evolving variables.
Thus far, I have only non-dimensionalized one of my thermodynamic variables (temperature),
and I will need to do another one later, but I won't worry about that for now.

\subsection{Non-dimensionalizing the energy equation}
In the spirit of the last section, I will now replace all of my dimension-filled variables
with dimensionless variables.  I have also replaced the viscous heating term with a term
that has equivalent units ($\mu \grad^2 \bm{u}^2$), to make the nondimensionalizing clearer.
For reasons that will become clear, I will leave $\rho$ filled with dimensions:
\begin{equation}
\rho c_v \frac{\bar{T}}{\bar{t}}\left(\frac{D T^*}{D t^*} + (\gamma - 1)T^*\grad^*\cdot\bm{u}^*\right)
- \frac{\kappa \bar{T}}{\bar{L}^2} (\grad^*)^2 T^* = - \frac{\mu}{\bar{t}^2}(\grad^*)^2 (u^*)^2
+ \kappa H
\end{equation}
Multiplying the full equation by $\bar{T}/ (\bar{t} \rho c_V)$, and then noting that
$\kappa / \rho = \chi$ and $\mu / \rho = \nu$ (the corresponding diffusivities), we get
\begin{equation}
\frac{D T^*}{D t^*} + (\gamma - 1)T^*\grad \cdot\bm{u}^* - \frac{\chi \bar{t}}{c_V \bar{L}^2}
(\grad^*)^2 T^* = -\frac{\nu}{c_V\bar{T}\bar{t}}(\grad^*)^2(u^*)^2 + \frac{\chi \bar{t}}{c_V\bar{T}} H
\end{equation}
At this point, I think it's logical to make a standard choice for $\bar{t}$,
$$
\bar{t} = c_V\frac{\bar{L}^2}{\chi},
$$
which is just the thermal diffusion timescale over the characteristic length scale of the
problem (which has yet to be specified).  With that in mind, and plugging in $\bar{t}$
throughout, we get
\begin{equation}
\frac{D T^*}{D t^*} + (\gamma - 1) T^* \grad^*\cdot\bm{u}^* - (\grad^*)^2T^* =
-\frac{\nu\chi}{c_V^2 \bar{T} \bar{L}^2} (\grad^*)^2(u^*)^2 + \frac{\bar{L}^2}{\bar{T}} H.
\end{equation}
At this point, the next step is not entirely obvious to me (I know it must be done to the
$H$ term, though).  I think there's either two ways to go:
\begin{enumerate}
\item The classic choice in Rayleigh-Benard: Non-dimensionalize the internal heating term
so that it is 1.
\item A more appropriate one for us (maybe?): non-dimensionalize the internal heating term
so that it is $\epsilon$, as this will reflect the size of thermo fluctuations that are produced.
I'll try that.
\end{enumerate}
So with choice number 2,
$$
\frac{\bar{L}^2}{\bar{T}} H  = \epsilon,
$$
and
$$
\bar{T} = \bar{L}^2 \frac{H}{\epsilon}
$$
Then, plugging that in, and dropping the stars because they're annoying, we're left with
a dimensionless equation that looks like
\begin{equation}
\boxed{
\frac{D T}{D t} + (\gamma - 1) T \grad\cdot\bm{u} - \grad^2 T 
= -\epsilon \frac{\nu\chi}{c_V^2 \bar{L}^4 H} \grad^2 u^2 + \epsilon}.
\end{equation}
Two interesting things to note here:
\begin{enumerate}
\item It's clearly visible what the scale of the internal heating is, and how it feeds into
the total temperature profile (this is some order $\epsilon$ deviation from the adiabatic).
\item The term in front of viscous heating is like Ra$^{-1}$, or something similar, which
means that the viscous heating term is shrinking linearly with the Rayleigh number.  This
will happen without the internal heating, too, it just is mega clear, here, and I've never
seen it before, neat.
\item This equation is really quite clean, and from it we have constrained two of our problem's
scales: \emph{time} and \emph{temperature}.  There are still two remaining free scales that
must be specified: \emph{length} and \emph{another thermodynamic variable}, but both of
these are bound to the choices that we've made here.
\end{enumerate}

\subsubsection{Sanity check: entropy equation}
As a sanity check, I want to do the same thing as I did above, but with the entropy
form of the energy equation from \cite{lecoanet&all2014}.  This equation takes the form:
\begin{equation}
\rho T \frac{D S}{D t} = -\grad \cdot (-\kappa \grad \frac{S}{c_P}) - \mu \grad^2 u^2 + \kappa H 
\end{equation}
where here, $\kappa = \chi_s \rho T$.  Note that $\mu = \rho \nu$.  The alternative form of this equation is one in which
the diffusion is $-\kappa \grad T$, where $\kappa = \chi_T \rho$.

If we assume constant kappa/mu, then divide by $\rho T$, we get
\begin{equation}
\frac{DS}{Dt} = \chi_S\grad^2 \frac{S}{c_P} - \frac{\nu}{T}\grad^2 u^2 + \chi_s H
\end{equation}
Doing the thing where we pull dimensions out of everything,
$$
\frac{\bar{S}}{\bar{t}}\frac{D S^*}{D t^*} = \chi_S \frac{\bar{S}}{c_P\bar{L}^2}(\grad^*)^2 S^*
- \frac{\nu}{\bar{T}\bar{t}^2}\frac{1}{T^*}(\grad^*)^2(u^*)^2 + \chi_s H
$$
Then, multiplying by $\bar{t}/\bar{S}$,
\begin{equation}
\frac{DS^*}{Dt^*} = \frac{\chi_S \bar{t}}{c_P \bar{L}^2}(\grad^*)^2 S^* 
- \frac{\nu}{\bar{T}\bar{S}\bar{t}}(\grad^*)^2(u^*)^2 + \frac{\chi_s \bar{t}}{\bar{S}} H
\end{equation}
So it's pretty clear that, again, we're going to non-dimensionalize time on the entropy
diffusion timescale,
$$
\bar{t} = \frac{\bar{L}^2}{\chi_S}c_P.
$$
And with that choice, it seems that the appropriate non-dimensionalization of entropy
is
$$
\bar{S} = \frac{\bar{t}\chi_S H}{\epsilon} = \frac{\bar{L}^2 H c_P}{\epsilon}
$$
Under these choices, the entropy energy equation with entropy diffusion becomes (dropping the
annoying stars because at this point all important things are non-dimensionalized),
\begin{equation}
\frac{D S}{D t} = \grad^2 S - \left(\frac{\nu\chi_S\epsilon}{\bar{T}\bar{L}^4 H c_P}\right)
\grad^2 u^2 + \epsilon
\end{equation}
Once again, viscous heating has a 1/Ra type term in front of it, with a few extra constants
thrown in.  

\subsection{Getting to a linearized momentum equation}
OK, so in order to do anything meaningful with the momentum equation (or at least, anything
that is both easy and meaningful), we need to change its form.  The first step in this process
is to decompose all thermodynamic variables into background and fluctuating components, e.g.
\begin{equation}
\begin{split}
\rho &= \rho_0 + \rho_1 \\
T &= T_0 + T_1 \\
P &= P_0 + P_1
\end{split}
\end{equation}
Next, I will make the assumption that my background (0) state is in hydrostatic equilibrium,
such that $-\grad P_0 - \rho g \hat{z} = 0$, and in making that assumption I will remove it
from the momentum equation.  This leaves me with
\begin{equation}
\rho \frac{D\bm{u}}{Dt} = -\grad P_1 - \rho_1 g \hat{z} - \grad\cdot\stressT.
\end{equation}
Next, I divide by $\rho$, and I put the stress tensor term into a more familiar form
$-\grad\cdot\stressT \sim -\mu \grad^2 \bm{u}$, so that I can more clearly see its
units and non-dimensionalize.  I also note that $\nu = \mu / \rho$.  Thus, my momentum
equation becomes
\begin{equation}
\frac{D \bm{u}}{Dt} = -\frac{\grad P_1}{\rho} + \frac{\rho_1}{\rho}\bm{g} - \nu \grad^2 \bm{u}.
\end{equation}
At this point, aside from the sort of contrived form of the viscous term, this equation is
still general to perturbations of any size.  But...now I'm going to make an assumption. All
thermodynamic perturbations are small compared to the background ($\rho_1 \ll \rho_0$,
and so on, such that $\rho = \rho_0 + \rho_1 \approx \rho_0$).

Under this approximation, we must examine the equation of state,
\begin{equation}
\frac{\grad S}{c_P} = \frac{1}{\gamma}\grad\ln(P_0 + P_1) - \grad \ln (\rho_0 + \rho_1).
\end{equation}
To make this prettier, I will do a couple of things.  First, note that
$$
\ln (A + B) = \ln\left(A\left[1 + \frac{B}{A}\right]\right) = \ln(A) + \ln\left(1 + \frac{B}{A}\right).
$$
Thus, we write the equation of state (after trivially integrating away the dels):
\begin{equation}
\frac{S_0}{c_P} + \frac{S_1}{c_P} = \frac{1}{\gamma}\ln(P_0) + \frac{1}{\gamma}\ln\left(1 + \frac{P_1}{P_0}\right)
- \ln(\rho_0) - \ln\left(1 + \frac{\rho_1}{\rho_0}\right).
\end{equation}
By definition, the ``0'' terms on both sides of the equation are equal to one another, so they
drop out.  Then, I use the (taylor expansion) relation that
$$
\ln(1 + x) \approx x
$$
for $x \ll 1$, and I find the linearized equation of state,
\begin{equation}
\boxed{
\frac{S_1}{c_P} = \frac{1}{\gamma} \frac{P_1}{P_0} - \frac{\rho_1}{\rho_0}
}.
\end{equation}

Now this term can go into the $\rho_1/\rho \bm{g} \approx \rho_1/\rho_0 \bm{g}$ term 
of the momentum equation, and I find
$$
\frac{\rho_1}{\rho_0}\bm{g} = \frac{1}{\gamma}\frac{P_1}{P_0}\bm{g} - \frac{S_1}{c_p}\bm{g}.
$$
Now, I play math tricks, and note that:
$$
\frac{1}{\gamma}\frac{P_1}{P_0}\frac{\rho_0}{\rho_0}\bm{g} = \frac{1}{\gamma}\frac{P_1}{\rho_0}\grad\ln P_0
= \frac{P_1}{\rho_0}\left(\frac{\grad S_0}{c_P} + \grad \ln \rho_0\right)
$$
where I've used the hydrostatic equilibrium of the initial conditions to cancel terms 
($\grad \ln P_0 = \rho_0 \bm{g} / P_0$), and then I've used the initial conditions
and their entropy EOS to do the next term ($\grad \ln P_0 / \gamma = \grad S_0 / c_P + \grad \ln \rho_0$).
Putting this into the momentum equation,
\begin{equation}
\frac{D\bm{u}}{Dt} = -\frac{\grad P_1}{\rho_0} + \frac{P_1}{\rho_0}\grad\ln\rho_0 + \frac{P_1}{\rho_0}\frac{\grad S_0}{c_P}
- \frac{S_1}{c_P}\bm{g} - \nu \grad^2\bm{u}
\end{equation}
The first two terms on the RHS condense, because
$$
\grad\left(\frac{P_1}{\rho_0}\right) = \frac{\grad P_1}{\rho_0} - \frac{P_1}{\rho_0}\grad\ln\rho_0
$$
and then I define $\varpi \equiv P_1/\rho_0$, such that the momentum equation is
\begin{equation}
\boxed{
\frac{D\bm{u}}{Dt} = -\grad \varpi + \varpi \frac{\grad S_0}{c_P} - \frac{S_1}{c_P}\bm{g} - \nu\grad^2\bm{u}.
}
\end{equation}
This form of the equation is ripe to be non-dimensionalized, but it is \emph{not} the full equation,
as the viscous term has been butchered so that it's dimensionality is clear, and we have assumed that
the fluctuations are small.

\subsection{Non-dimensional, linear momentum equation}
Taking the linearized momentum equation and pulling out dimensions from all terms except those that
have $\varpi$,
$$
\frac{\bar{L}}{\bar{t}^2}\frac{D \bm{u}^*}{D t^*} = \left(-\grad\varpi + \varpi\frac{\grad S_0}{c_P}\right)
+ \frac{\bar{S}g}{c_P} S_1^* \hat{z} - \frac{\nu}{\bar{L}\bar{t}}(\grad^*)^2\bm{u}^*.
$$
If we multiply through by $\bar{t}^2/\bar{L}$, we get
$$
\frac{D\bm{u}^*}{Dt^*} = \frac{\bar{t}^2}{\bar{L}}\left(-\grad\varpi + \varpi\frac{\grad S_0}{c_P}\right)
+ \frac{\bar{S} g \bar{t}^2}{\bar{L} c_P} S_1^* \hat{z} - \frac{\nu \bar{t}}{\bar{L}^2}(\grad^*)^2\bm{u}^*.
$$
The last two terms are the ones we're interested in now.  On the last term, after plugging in the
definition of $\bar{t}$ from the energy equation, we get $\nu c_V / \chi$, which is basically
Pr.  On the second to last term, we get $\bar{S} \bar{L}^3 c_V^2 g/ c_P \chi^2$, which is basically
RaPr.  So, under this set of equations,
\begin{equation}
\text{Pr} \equiv c_V \frac{\nu}{\chi} \qquad \text{Ra} = c_V\frac{g \bar{L}^3 (\bar{S}/c_P)}{\chi \nu}
\end{equation}
Which is to say, these are the standard definitions of Ra and Pr, but with $c_V$'s in front,
which means either I've messed up here, or I've been careless in the past.

Either way, the important thing to do now is figure out $\bar{L}$ and $\bar{S}$.

\subsubsection{Argument for $\bar{L}$}
I think this one is simple.  The length scale we're interested in is the length scale of the
system in which a superadiabatic flux is being carried -- this is the scale over which we 
\emph{know} we need convection, and for our setup, as I explained in our ``paper,''
$$
\bar{L} = d_{\text{conv}} = \frac{\epsilon}{H c_P},
$$
where the heating term's magnitude is
$$
H = \frac{\epsilon}{L_z c_P (1 - f)},
$$
which means that in terms of control parameters, as we would anticipate,
$$
\bar{L} = (1 - f) L_z.
$$

\subsubsection{Argument for $\bar{S}$}
This one's trickier.  I think it's important to remember that an underlying assumption of
this whole form of the rayleigh number and prandtl number is that we used the linearized equation
of state to get here.  So in a consistently scaled system,
$$
\frac{\bar{S}}{c_P} \sim \frac{1}{\gamma}\frac{\bar{T}}{T_{0, scale}} - \frac{\gamma-1}{\gamma}\frac{\bar{\rho}}{\rho_{0, scale}}.
$$
Or something of the sort.  So here, I'm explicitly stating that $\bar{T}$ is non-dimensionalizing the
\emph{non-background} motions, and the same goes for $\bar{S}$, and the same goes for $\bar{\rho}$. 
This is actually an important distinction. I think I need to use something like eqn (14) of
\cite{brown&all2012} along with this assumption to get a reasonable scale for $S_0$
(and I'll need to bring in my assumptions about $\bar{t}$ and $\bar{L}$ to get there.)

The linearized momentum equation \citep{brown&all2012} is
\begin{equation}
\frac{\partial \rho_1}{\partial t} + \bm{u}\cdot\grad\rho_0 = -\rho_0 \grad\cdot\bm{u}
\end{equation}
and I just realized that this is going to give us just $\rho_1/\rho_0 = $ whatever I set
it to be, so oops.

But still, let's just say that $T_0, \rho_0$ are O(1), then
$$
\frac{\bar{S}}{c_P} \sim \frac{1}{\gamma}\bar{T} - \frac{\bar{\rho}}{c_P}.
$$
We have $\bar{T}$ from the energy equation,
$$
\bar{T} = \bar{L}^2 H,
$$
so ok.

It's a bit reckless, but for now, a \emph{good approximation} is that
\begin{equation}
\bar{S} = c_P(\bar{L}^2 H).
\end{equation}

\subsection{Making a Rayleigh Number}
So in the end, our initial Rayleigh number (to get some first cut simulations up and running)
is going to be
\begin{equation}
\boxed{
\text{Ra} = \frac{g \bar{L}^3 (\bar{S}/c_P)}{\nu\chi} = \frac{g (1 - f)^5 L_z^5 H}{\nu\chi} = 
\frac{g (1 - f)^4 L_z ^4 \epsilon }{\nu\chi}}
\end{equation}


\bibliography{../biblio.bib}
\end{document}

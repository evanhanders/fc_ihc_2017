%\documentclass[iop]{emulateapj}
\documentclass[aps, pre, onecolumn, nofootinbib, notitlepage, groupedaddress, amsfonts, amssymb, amsmath, longbibliography]{revtex4-1}
\usepackage{graphicx}
\usepackage{hyperref}
\usepackage{xcolor}
\hypersetup{
    colorlinks,
    linkcolor={red!50!black},
    citecolor={blue!50!black},
    urlcolor={blue!80!black}
}
\usepackage{bm}
\usepackage{natbib}
\usepackage{longtable}
\LTcapwidth=0.87\textwidth

\newcommand{\Div}[1]{\ensuremath{\nabla\cdot\left( #1\right)}}
\newcommand{\angles}[1]{\ensuremath{\left\langle #1 \right\rangle}}
\newcommand{\grad}{\ensuremath{\nabla}}
\newcommand{\RB}{Rayleigh-B\'{e}nard }
\newcommand{\stressT}{\ensuremath{\bm{\bar{\bar{\Pi}}}}}
\newcommand{\lilstressT}{\ensuremath{\bm{\bar{\bar{\sigma}}}}}
\newcommand{\nrho}{\ensuremath{n_{\rho}}}
\newcommand{\approptoinn}[2]{\mathrel{\vcenter{
	\offinterlineskip\halign{\hfil$##$\cr
	#1\propto\cr\noalign{\kern2pt}#1\sim\cr\noalign{\kern-2pt}}}}}

\newcommand{\appropto}{\mathpalette\approptoinn\relax}

\newcommand\mnras{{MNRAS}}%

\begin{document}
\author{Evan H. Anders}
\affiliation{Dept. Astrophysical \& Planetary Sciences, University of Colorado -- Boulder, Boulder, CO 80309, USA}
\affiliation{Laboratory for Atmospheric and Space Physics, Boulder, CO 80303, USA}
\author{Benjamin P. Brown}
\affiliation{Dept. Astrophysical \& Planetary Sciences, University of Colorado -- Boulder, Boulder, CO 80309, USA}
\affiliation{Laboratory for Atmospheric and Space Physics, Boulder, CO 80303, USA}
\title{Internally heated, stratified, compressible convection}

\begin{abstract}
An abstract will go here eventually
\end{abstract}
\maketitle


\section{Introduction}
\label{sec:intro}
\begin{enumerate}
\item Importance of internal heating in natural convective systems
\item Shape of internal heating in something like Sun
\item Studies in Rayleigh-Benard
\item Goals of this paper:
\begin{enumerate}
\item Show how to set up internally heated polytropes, with stable layers (or without)
\item Show how the resultant stratification depends on combo of lower flux + internal heating profile choices
\item Show how stable layer affects surface power spectrum
\end{enumerate}
\end{enumerate}

\section{Experiment} 
\label{sec:experiment}

\subsection{General system info and temperature profile}
We choose a parameter, $\epsilon$, which sets the initial scale of gravity,
\begin{equation}
g \equiv m_{ad} + 1 - \epsilon,
\end{equation}
where $m_{ad} \equiv (\gamma - 1)^{-1}$, and $\gamma = c_P / c_V = 5/3$.  The atmosphere
has a constant internal heating term, such that the energy equation looks like
\begin{equation}
\frac{\partial}{\partial t}\left(\left[\rho\frac{|\bm{u}|^2}{2} + c_v T + \phi\right]\right)
+ \grad\cdot\left(\bm{F}_{\text{conv}} + \bm{F}_{\text{cond}}\right) = \kappa H,
\end{equation}
where $H$ is an internal heating term.  In the initial, static, conductive state, the
conductive flux must balance the internal heating term,
\begin{equation}
\grad\cdot\left(-\kappa\grad T_{c}\right) = \kappa H,
\end{equation}
A useful form of the initial temperature gradient in such a system is
\begin{equation}
\boxed{
\grad T_{c} = - 1 + H (L_z - z)} .
\end{equation}
This initial conductive solution is \emph{not} the adiabatic solution to this atmosphere,
\begin{equation}
\grad T_{ad} = -\frac{g}{c_P} = -1 + \frac{\epsilon}{c_P}.
\end{equation}
If we integrate the conductive solution, our initial temperature profile is
$$
T_c(z) = (H L_z - 1) z - \frac{H}{2}z^2 + C
$$
and if we want $T_c(L_z) = T_t$,
\begin{equation}
\boxed{
T_c(z) = -\frac{H}{2}z^2 - (1 - H L_z) z + \left(T_t - \frac{H}{2}L_z^2 + L_z\right)}.
\end{equation}
So our initial temperature gradient is now quadratic.

The initial \emph{superadiabatic temperature gradient} is an important quantity, as it
determines whether or not the atmosphere is stable to convection.  This gradient
in these systems is
\begin{equation}
\grad T_{c} - \grad T_{ad} = H(L_z - z) - \frac{\epsilon}{c_P}.
\end{equation}
When this quantity is zero, the atmosphere is carrying the maximum amount of flux
that conductivity can carry.  When this value is less than zero, the atmosphere
requires convection to carry the flux.  Solving for a generalized zero-point, we
find that
\begin{equation}
z_{\text{cross}} \equiv L_z - \frac{\epsilon}{H c_P}.
\end{equation}
In order for the atmosphere to have any region that is unstable to convection,
we require that $0 < \epsilon / ( H c_P ) < L_z$.  In the case that 
$\epsilon / ( H c_P ) < L_z$, the whole atmosphere is unstable to convection.
Further, the anticipated length scale of the convecting region of the atmosphere
is the full atmosphere above $z_{\text{cross}}$, or
\begin{equation}
d_{\text{conv}} = L_z - z_{\text{cross}} = \frac{\epsilon}{H c_P}.
\end{equation}
It's maybe not super important, but the temperature at the base of the convection zone is
$$
T_c(z_{\text{cross}}) = -\frac{H}{2}z_{\text{cross}}^2 - (1 - HL_z)z_{\text{cross}}
+ \left(T_t - \frac{H}{2}L_z^2 + L_z\right) = T_t + d_{\text{conv}} - \frac{H}{2} d_{\text{conv}}^2.
$$
The temperature gradient there is
$$
\grad T_c(z_{\text{cross}}) = -\left(1 - \frac{\epsilon}{c_P}\right) = \grad T_{\text{ad}}
$$

I think there are two natural control parameters of this system:
\begin{enumerate}
\item $f \equiv z_{\text{cross}} / L_z$, the fraction of the atmosphere that is stable
($f = 0$ means the full atmosphere convects, $f = 0.5$ means that the upper half convects, etc).
This \emph{should} be the parameter that determines the depth of the stable layer.
\item $\epsilon$, the deviation of the initial temperature gradient (at the top of the atmosphere)
from the adiabatic temperature gradient. This \emph{should} be the parameter that determines
the depth of the stable layer.
\end{enumerate}
By specifying these two parameters, we retrieve the magnitude of the internal heating,
$ f = 1 - \epsilon / (H c_P L_z)$
\begin{equation}
\boxed{
H = \frac{\epsilon}{L_z c_P (1 - f)} = \frac{\epsilon}{d_{\text{conv}}c_P}}.
\end{equation}

\subsection{Initial density stratification}

The initial density profile is set by hydrostatic equilibrium,
\begin{equation}
\grad\ln\rho_c = -\frac{g + \grad T_c}{T_c} = -\grad\ln T_c- \frac{g}{T_c}.
\end{equation}
If we integrate over this function, the first term is super easy, and we end up having
\begin{equation}
\ln\rho_c = - \ln T_c - \int\frac{g}{T_c}dz + \xi,
\end{equation}
where $\xi$ is a constant.  The next integral is harder.  In general, this integral has the form
\begin{equation}
\int \frac{1}{Az^2 + Bz + C}dz = \frac{2}{\sqrt{4 AC - B^2}} \tan^{-1}\left(\frac{2 Az + B}{\sqrt{4 AC - B}}\right).
\end{equation}
For our specific implementation, we get our constants from the temperature profile,
$$
A = -H / 2 \qquad B = (HL_z - 1) \qquad C = \left(T_t - \frac{H}{2}L_z^2 + L_z\right),
$$
and note that $2Az + B = -Hz + (HL_z - 1) = \grad T_c$.  Note also that
$4AC - B^2 = -2H(T_t - (H/2)L_z^2 + L_z) - (H^2 L_z^2 - 2HL_z + 1) = -2HT_t - 1$,
so $\sqrt{4AC - B^2} = i\xi$, where $i = \sqrt{-1}$ and $\xi = \sqrt{1 + 2HT_t}$.
With these definitions in hand, we find that
\begin{equation}
\ln\rho_c = -\ln T_c + \frac{2 g i}{\xi}\tan^{-1}\left(-i\frac{\grad T_c}{\xi}\right)
\end{equation}
Now it's important to know that there's a neat identity: $\tan^{-1}(-i A) = -i\tanh^{-1}(A)$,
and plugging this in we get
\begin{equation}
\ln\rho_c = -\ln T_c + \frac{2 g}{\xi}\tanh^{-1}\left(\frac{\grad T_c}{\xi}\right).
\end{equation}
At this point, there's another useful identity: $\tanh^{-1}(A) = (1/2)\ln([1 + A]/[1 - A])$,
so the density profile is
\begin{equation}
\ln\rho_c = -\ln T_c + \frac{g}{\xi}\ln\left(\frac{\xi + \grad T_c}{\xi - \grad T_c}\right) + \Xi.
\end{equation}
In general, this profile is valid for any value of $\rho_t$, such that
\begin{equation}
\Xi = \ln\rho_t + \ln T_t - \frac{g}{\xi}\ln\left(\frac{\xi - 1}{\xi + 1}\right),
\end{equation}
where $\grad T_c(L_z) = -1$ has already been previously specified.  This means that the general,
full form of the log density is
\begin{equation}
\boxed{
\ln\left(\frac{\rho_c}{\rho_t}\right) = -\ln\left(\frac{T_c}{T_t}\right) + 
\frac{g}{\xi}\ln\left(\frac{\xi + \grad T_c}{\xi - \grad T_c}\cdot\frac{\xi + 1}{\xi - 1}\right) }.
\end{equation}

\subsubsection{Specifying the depth of the atmosphere}
I'm going to define a different way of specifying the whole atmospheric depth.  I still like what
I have in the section below for finding the depth for finding the depth of the convection zone.
However, after that point, the choice of $f$ is opaque and hard to understand.  Thus, I'll
define a new parameter,
\begin{equation}
r \equiv \frac{L_{RZ}}{L_{CZ}},
\end{equation}
such that as $r \rightarrow 0$, we have an atmosphere which is entirely convective, and
as $r \rightarrow \infty$, we have an atmosphere which is entirely a radiative zone.

If the depth of the radiative zone is known, this means that the depth of the total
atmosphere is just
\begin{equation}
\boxed{
L_z = L_{CZ}(1 + r)}.
\end{equation}
A few points of comparison:
\begin{enumerate}
\item $f = 0$ was the same thing as $r = 0$.
\item $f = 0.5$ is now $r = 1$.
\item $f = 0.75$ is now $r = 3$ (RZ is 3x deeper than CZ)
\item $f = 0.9$  is now $r = 9$.  
\end{enumerate}
Basically the important thing to note is that the range $f = 0 \rightarrow 0.5$ basically didn't
change much, but the range $f = 0.5 \rightarrow 1$ changed things hugely in a hard to understand,
nonlinear way.  Now, changes of $r$ correspond to understandable changes in the atmosphere.

\subsubsection{Old version of atmospheric depth specification}
I still want $n_\rho$ to be the parameter that specifies the depth of the parameter,
and I specifically want it to specify the number of density scale heights in the part of the
atmosphere that carries a superadiabatic flux.  

To get started, I need to consider the value of $\ln\rho_c$ at the bottom of the atmosphere,
$$
\ln\rho_c(z=0) = n_\rho = -\ln\left(\frac{T_c(z=0)}{T_t}\right)
+ \frac{g}{\xi}\ln\left(\frac{(\xi + \grad T_c(z=0))(\xi + 1)}{(\xi - \grad T_c(z=0))(\xi - 1)}\right)
$$
And we have a problem here, right?  The stuff on the RHS is a complex function of $\epsilon$ and
$L_z$, and we're trying to specify $L_z$ using $n_\rho$, not the other way around.
I'm going to consistently move all of the terms from the RHS to the LHS, then
non-dimensionalize and assume $T_t = \rho_t = 1$, making the function
\begin{equation}
f(L_{z}) = n_\rho + \ln\left(T_t - \frac{H}{2}L_z^2 + L_z\right) - 
\frac{g}{\xi}\ln\left(\frac{\xi^2 - 1 + HL_z(\xi + 1)}{\xi^2 - 1 - HL_z(\xi - 1)}\right).
\end{equation}
When this function is zero, $L_z$ appropriately captures the right number of density scale heights.

Since we have set up all of our non-dimensionalizations at the \emph{top} of the atmosphere, and we
are building atmospheres which have CZs above RZs, we can actually really simply specify the
density stratification of the CZ.  The CZ is an atmosphere with $f = 0$ whose $z = 0$ starts
at $z_{cross}$.  If we set $n_\rho = n_{\rho, cz}$ and set $f = 0$, and then find the appropriate
$L_z$ from $f(L_z)$, that will give us $L_{z-cz}$.  And since that is a \emph{depth} from
the top of the atmosphere, we can use that depth to determine how big the rest of the
atmosphere should be for our ``true'' value of $f$. That is,
\begin{equation}
L_z = \frac{L_{z-cz}}{1 - f}.
\end{equation}
This way, by adding the RZ, we're actually extending the atmosphere captured down into the RZ,
rather than gobbling up part of the CZ with the RZ.  In order to find $L_{z-cz}$, a
numerical root-finding algorithm is the logical choice.

\subsection{Entropy profile}
Ok, the specific entropy profile is defined according to the standard entropy equation
\begin{equation}
\frac{\grad S}{c_P} = \frac{1}{\gamma}\grad\ln P - \grad \ln \rho = \frac{1}{\gamma}\grad\ln T - \frac{\gamma - 1}{\gamma} \grad\ln\rho
\end{equation}
This means that, in the conductive state, where $\grad \ln \rho_c = -g/T_c - \grad\ln T_c$,
$$
\frac{\grad S_c}{c_P} = \frac{1}{\gamma}\grad\ln T_c - \frac{\gamma-1}{\gamma}\left(-\frac{g}{T_c} - \grad\ln T_c\right)
= \grad\ln T_c + \frac{\gamma - 1}{\gamma}\frac{g}{T_c}
$$
Which means that the entropy gradient here, and the integration of it, is strikingly similar to
what we had to do to integrate the log density profile.  Actually it's basically exactly the same, but with
some extra constants in front and also with a negative sign flip.  So we have
$$
\frac{S_c}{c_P} = \ln T_c - \frac{\gamma-1}{\gamma}\frac{g}{\xi}\ln\left(\frac{\xi + \grad T_c}{\xi - \grad T_c}\right) + Y
$$
At this point we're free to choose any value of $S_c$ that we want for the gauge.  And really, it's arbitrary.  So I'm
going to set the entropy at the top of the atmosphere to be zero to just get an analytical nice expression, and I find
$$
Y = \frac{\gamma - 1}{\gamma} \frac{g}{\xi} \ln \left(\frac{\xi - 1}{\xi + 1}\right) - \ln T_t,
$$
which means the entropy profile is
\begin{equation}
\boxed{
\frac{S_c}{c_P} = \ln\left(\frac{T_c}{T_t}\right) 
- \frac{\gamma-1}{\gamma}\frac{g}{\xi}\ln\left(\frac{(\xi + \grad T_c)(\xi + 1)}{(\xi - \grad T_c)(\xi - 1)}\right)}.
\end{equation}
And, since we have non-dimensionalized this system such that $S = 0$ at the top of the atmosphere,
then the initial CONVECTIVE ZONE entropy jump across the conductive atmosphere is, assuming
$T_t = 1$,
\begin{equation}
\frac{\Delta S_c}{c_P} = \ln\left(1 + d_{\text{conv}} - \frac{H}{2} d_{\text{conv}}^2\right)
+ \frac{\grad T_{ad}}{\xi}\ln\left(\frac{(1 + \grad T_{ad}/\xi) (1 + \xi)}{(1 - \grad T_{ad}/\xi)(\xi - 1)}\right)
\end{equation}
which is still frankly pretty horrifying.  I think at this point I need to remember that
$$
\xi = \sqrt{1 + 2H} \approx 1 + H,
$$
where the last term is a linear Taylor expansion around $2H = 0$.  So this is a good approximation
for small $H$, which is honestly generally a good approximation ($H = \epsilon / (d_{\text{conv}}c_P)$).
So we have actually (before I went and complicated things in the log),
$$
\frac{\Delta S_c}{c_P} = \ln\left ( 1 + d_{\text{conv}} - \frac{H}{2}d_{\text{conv}^2}\right)
+ \frac{\grad T_{ad}}{1 + H}\ln\left(\frac{(1 + H + \grad T_{ad})(2 + H)}{(1 + H - \grad T_{ad})H}\right)
$$
and $\grad T_{ad} = -(1 - \epsilon / c_P) = -(1 - H d_{\text{conv}})$, so
$$
\frac{\Delta S_c}{c_P} = \ln\left ( 1 + d_{\text{conv}} - \frac{H}{2}d_{\text{conv}^2}\right)
+ \frac{-(1 - Hd_{\text{conv}})}{1 + H}\ln\left(\frac{(1 + d_{\text{conv}})(2 + H)}{(2 + H[1 - d_{\text{conv}}])}\right)
$$
OK, now we take advantage of the assumption we already made, $H \ll 1$, so we get
$$
\frac{\Delta S_c}{c_P} = \ln\left ( 1 + d_{\text{conv}} - \frac{H}{2}d_{\text{conv}^2}\right)
- (1 - Hd_{\text{conv}})\ln\left(\frac{(1 + d_{\text{conv}})2}{(2 - Hd_{\text{conv}})}\right)
$$
and since $\ln(A/B) = \ln(A) - \ln(B)$,
$$
\frac{\Delta S_c}{c_P} = \ln\left ( 1 + d_{\text{conv}} - \frac{H}{2}d_{\text{conv}^2}\right)
- (1 - Hd_{\text{conv}})\left[\ln(1 + d_{\text{conv}}) - \ln(1 - H d_{\text{conv}}/2)\right]
$$
and now we can start to group terms a bit differently,
$$
\frac{\Delta S_c}{c_P} = \ln \left(\frac{1 + d_{\text{conv}} - H d_{\text{conv}}^2 / 2}{1 + d_{\text{conv}}}\right)
+ H d_{\text{conv}}\ln(1 + d_{\text{conv}}) + (1 - H d_{\text{conv}})\ln(1 - H d_{\text{conv}}/2)
$$
Yikes. At this point, we have to make one more assumption, which is a little less \emph{always}
true, but it still fits into the assumption for the linearized EOS.  That assumption is that
$d_{\text{conv}}H \ll 1$, such that we can use the $\ln( 1 + x ) \approx x$ for small $x$ assumption.
Then, we get
$$
\frac{\Delta S}{c_P} \approx -\frac{H d_{\text{conv}}^2}{2(1 + d_{\text{conv}})} + H d_{\text{conv}}^2
+ (1 - H d_{\text{conv}})\left(-\frac{H d_{\text{conv}}}{2}\right)
$$
or,
$$
\frac{\Delta S}{c_P} \approx -\frac{H d_{\text{conv}}^2}{2(1 + d_{\text{conv}})} + H d_{\text{conv}}^2
 - \frac{1}{2}H d_{\text{conv}} + \frac{H^2 d_{\text{conv}}^2}{2}
$$
or,
$$
\frac{\Delta S}{c_P} \approx \frac{-H d_{\text{conv}}^2 - H d_{\text{conv}} + (H + H^2)d_{\text{conv}}^2 + (2H + H^2)d_{\text{conv}^3}}{2(1 + d_{\text{conv}})}
= H d_{\text{conv}}\frac{-1  + H d_{\text{conv}} + (2 + H)d_{\text{conv}}^2}{2(1 + d_{\text{conv}})}
\sim H d_{\text{conv}}^2.
$$
And, for the record, that's about the order of magnitude I was anticipating it to be, and this is about the
same order of magnitude as we expect the perturbations in $T$ and $\rho$ to be.  Sweet!

Unfortunately though, just doing some numerical tests, it looks like $\Delta S_c/c_P$ actually scales more like
$d_{\text{conv}}H$.  Which means either (A) I'm missing some really meaningful prefactors or (B) I messed up
somewhere.

\subsubsection{Alternate approach}
In an alternate universe, it's possible that I would prefer less self-abuse.  In that universe, I'd
recognize that
$$
\frac{S_c}{c_P} = \frac{1}{\gamma} \ln T_c - \frac{\gamma - 1}{\gamma} \ln \rho_c.
$$
And, thus, I'd see that
$$
\frac{\Delta S_c}{c_P} = -\frac{1}{\gamma}\ln T_c(z = z_{\text{cross}}) + \frac{\gamma - 1}{\gamma} n_{\rho}
= -\frac{1}{\gamma}\ln\left(1 + d_{\text{conv}} - \frac{H}{2}d_{\text{conv}^2}\right) + \frac{n_{\rho}}{c_P}
$$
and since $\ln(A + B) = \ln(A(1 + B/A)) = \ln(A) + \ln(1 + B/A)$ we can break up the first term as
$$
\frac{\Delta S_c}{c_P} = -\frac{1}{\gamma}\ln(1 + d_{\text{conv}}) - \frac{1}{\gamma}\ln\left(1 - \frac{(H/2)d_{\text{conv}}^2}{1 + d_{\text{conv}}}\right)
+ \frac{n_\rho}{c_P}
$$
At this point, I'm going to make an assumption.  It's not quite true for large $\epsilon$.  But,
$\ln(1 + L_z) \approx n_{\rho} / (m_{ad} - \epsilon)$.  This is exactly true for polytropes, but
not for these systems.  Either way,
$$
\frac{\Delta S_c}{c_P} \approx - \frac{1}{\gamma}\ln\left(1 - \frac{(H/2)d_{\text{conv}}^2}{1 + d_{\text{conv}}}\right)
-\frac{n_{\rho}}{\gamma(m_{ad}-\epsilon)} + \frac{n_{\rho}}{c_P} = 
- \frac{1}{\gamma}\ln\left(1 - \frac{(H/2)d_{\text{conv}}^2}{1 + d_{\text{conv}}}\right) - \frac{\epsilon n_\rho}{c_P}\left(\frac{\gamma - 1}{\gamma - \epsilon(\gamma - 1)}\right)
$$
...and I'm kind of stuck again.  Anyways, the important thing I just learned here (from some numerics) is that
$\Delta S_c \approx -\epsilon n_{\rho}$, which kind of actually makes sense from the form of the equation.

\subsection{Thermal Equilibrium}
The next thing to ask is: is our conductive profile in thermal equilibrium?  If we have an energy equation of the
form:
\begin{equation}
-\kappa \grad^2 T_c = \kappa H,
\end{equation}
which is to say, great!  We have $\grad^2 T_c = -H$ in our setup.  To keep our atmosphere simple
to understand later down the line, we need to note that:
\begin{enumerate}
\item $\kappa$ should be constant with the depth of the atmosphere, which means that
$\chi = \kappa/\rho$ should be set based on our new initial conditions, which are not
the same as what we had in a polytrope, but it is similar.
\item $\kappa$ should be constant in time for easy analyses, which means that $\chi$ is free
to evolve as the density profile changes.
\end{enumerate}

\subsection{Pressure profile}
It's not mega important, but according to hydrostatic equilibrium,
\begin{equation}
\grad \ln P_c = -\frac{g}{T_c}.
\end{equation}
Cool.  So basically the pressure profile is just the complicated, gross part of the density
profile from above.

\subsection{Entropy budget \& boundary conditions}
I need to think carefully on this.  I think I may actually need to do fixed temperature
boundary conditions, but...I'm not sure.  I really need to think carefully.

I need to consider how much mass I have, and how my atmosphere is going to evolve in response
to the boundary conditions that I put on the edges of it.

\bibliography{../biblio.bib}
\end{document}
